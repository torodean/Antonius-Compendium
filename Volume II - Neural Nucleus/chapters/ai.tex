\chapter{Artificial Intelligence}
\thispagestyle{fancy}


\section{Introduction}

Artificial Intelligence (AI) is the field of study concerned with the design and development of systems capable of performing tasks that typically require human intelligence. These tasks include learning, reasoning, problem-solving, perception, and language understanding. In a sense, this is an attempt to artificially mimic human intelligence while simultaneously combining it with the advantages that modern technological computing powers bring. AI spans a wide range of sub-fields, from symbolic logic and knowledge representation to machine learning and neural networks. It intersects with disciplines such as computer science, mathematics, neuroscience, and philosophy. 

Modern AI systems are broadly categorized into two classes: \textit{narrow AI}, designed for specific tasks, and \textit{general AI}, which aspires to emulate human-level cognition across diverse domains. Key concepts include algorithms, data structures, optimization, statistical inference, and computational models of learning. Recent advancements in Large Language Models (LLMs) have demonstrated the ability of transformer-based architectures to generate coherent text, perform reasoning tasks, and interface with complex domains using natural language, which gives an appearance for the foundations of creating a \textit{general AI}. Artificial General Intelligence (AGI) refers to a type of AI that possesses the ability to understand, learn, and apply knowledge across a wide range of tasks at a level comparable to human intelligence, which is the goal of many.

Applications of AI are pervasive, influencing medicine, finance, robotics, language processing, and decision-making systems. Continued advancement in AI raises important technical, ethical, and societal questions, many of which remain open areas of research.
