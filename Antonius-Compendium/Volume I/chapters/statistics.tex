\chapter{Statistics and Probability}
\thispagestyle{fancy}

A \keyword{state} (or outcome) is particular condition that something is in at a specific time. 

\begin{defn}[System]{System}
A \keyword{system} is an activity, experiment, process, or model with states or outcomes that are typically subject to uncertainty.
\end{defn}

\begin{defn}[Sample Space]{Sample Space}
A \keyword{sample space} of an system is the set of all possible states of a system.
\end{defn}

\begin{defn}[Event]{Event}
An \keyword{event} (also may be referred to as a trial or measurement) is any subset or collection of states contained in the sample space of a system. An event is a \keyword{simple event} if it consists of exactly one state and a \keyword{compound event} if it consistst of more than one state.
\end{defn}

\begin{defn}[Probability]{Probability}
An \keyword{probability} $p$ can be defined as the asymptotic frequency of a system in the state $s$ ($s \in \Omega$, where $\Omega$ is the sample space of the system) by the total number of occurances of that state $N_s$ in the limit of an infinite number of events $N$.
    \begin{align}
        p(s) = \lim_{N\rightarrow\infty}\frac{N_s}{N} \\
	p(s) \in [0,1] \forall s \in \Omega
    \end{align}
For a system with $n$ states, the total probabilities of all states must normalize to one.
    \begin{align}
        \sum_{i=0}^{n}p(i) = \sum_{s \in \Omega}p(i) = \lim_{N\rightarrow\infty}\frac{1}{N}\sum_{i=0}^{n}N_i = 1
    \end{align}
A \keyword{Bayesian probability} is defined as a person's knowledge of the outcome of a trial, based on the evidence at their disposal - often encompanied by an associated error. A \keyword{model probability} is an assumption or guess for the probability given the possibility of an infinite nuber of trials.
\end{defn}

