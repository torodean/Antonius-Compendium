\newcommand{\andspace}[1]{\hspace{#1}\textrm{and}\hspace{#1}}

\numberwithin{equation}{section}
\setlength{\columnsep}{.5cm}
\setlength{\columnseprule}{1pt}
\def\columnseprulecolor{\color{black}}

\newcommand{\abs}[1]{\left| #1 \right|}
\newcommand{\inner}[1]{\langle #1 \rangle}
\newcommand{\norm}[1]{\left\lVert#1\right\rVert}
\newcommand{\spanvect}{\textnormal{span}}
\newcommand{\union}{\cup}
\newcommand{\Union}{\bigcup}

% Defines a keyword which will bold and add a word to the index.
\newcommand{\keyword}[1]{\textbf{#1}\index{#1}}

% Create a section without making the section title.
\newcommand\invisiblesection[1]{%
	\refstepcounter{section}%
	\addcontentsline{toc}{section}{\protect\numberline{\thesection}#1}%
	\sectionmark{#1}}

% Makes a chapter with no title
\makeatletter
\newcommand{\unchapter}[1]{%
	\begingroup
	\let\@makechapterhead\@gobble % make \@makechapterhead do nothing
	\chapter{#1}
	\endgroup
}
\makeatother




%%% This defines the solution environment for you to write your solutions
\newenvironment{soln}
{\let\oldqedsymbol=\qedsymbol
	\renewcommand{\qedsymbol}{$ $}
	\begin{proof}[\bfseries\upshape \color{blue}Derivation]\color{blue}}
	{\end{proof}
	\renewcommand{\qedsymbol}{\oldqedsymbol}}

\newenvironment{note}
{\let\oldqedsymbol=\qedsymbol
	\renewcommand{\qedsymbol}{$ $}
	\begin{proof}[\bfseries\upshape \color{red}Note]\color{red}}
	{\end{proof}
	\renewcommand{\qedsymbol}{\oldqedsymbol}}
	

\newenvironment{Deletion}
{\let\oldqedsymbol=\qedsymbol
	\renewcommand{\qedsymbol}{$ $}
	\begin{proof}[\bfseries\upshape \color{red}Deletion]\color{red}}
	{\end{proof}
	\renewcommand{\qedsymbol}{\oldqedsymbol}}


%theorem
\newcounter{theo}[section] \setcounter{theo}{0}
\renewcommand{\thetheo}{\arabic{chapter}.\arabic{section}.\arabic{theo}}
\newenvironment{theo}[2][]{%
	\refstepcounter{theo}%
	\ifstrempty{#1}%
	{\mdfsetup{%
			frametitle={%
				\tikz[baseline=(current bounding box.east),outer sep=0pt]
				\node[anchor=east,rectangle,fill=blue!20]
				{\strut Theorem~\thetheo};}}
	}%
	{\mdfsetup{%
			frametitle={%
				\tikz[baseline=(current bounding box.east),outer sep=0pt]
				\node[anchor=east,rectangle,fill=blue!20]
				{\strut Theorem~\thetheo:~#1};}}%
	}%
	\mdfsetup{innertopmargin=10pt,linecolor=blue!20,%
		linewidth=2pt,topline=true,%
		frametitleaboveskip=\dimexpr-\ht\strutbox\relax
	}
	\begin{mdframed}[]\relax%
		\label{#2}}{\end{mdframed}}
%%%%%%%%%%%%%%%%%%%%%%%%%%%%%%

%Lemma
\newcounter{lemm}[section] \setcounter{lemm}{0}
\renewcommand{\thelemm}{\arabic{chapter}.\arabic{section}.\arabic{lemm}}
\newenvironment{lemm}[2][]{%
	\refstepcounter{lemm}%
	\ifstrempty{#1}%
	{\mdfsetup{%
			frametitle={%
				\tikz[baseline=(current bounding box.east),outer sep=0pt]
				\node[anchor=east,rectangle,fill=green!20]
				{\strut Lemma~\thelem};}}
	}%
	{\mdfsetup{%
			frametitle={%
				\tikz[baseline=(current bounding box.east),outer sep=0pt]
				\node[anchor=east,rectangle,fill=green!20]
				{\strut Lemma~\thelem:~#1};}}%
	}%
	\mdfsetup{innertopmargin=10pt,linecolor=green!20,%
		linewidth=2pt,topline=true,%
		frametitleaboveskip=\dimexpr-\ht\strutbox\relax
	}
	\begin{mdframed}[]\relax%
		\label{#2}}{\end{mdframed}}
%%%%%%%%%%%%%%%%%%%%%%%%%%%%%%

%Proof
\newcounter{prf}[section]\setcounter{prf}{0}
\renewcommand{\theprf}{\arabic{chapter}.\arabic{section}.\arabic{prf}}
\newenvironment{prf}[2][]{%
	\refstepcounter{prf}%
	\ifstrempty{#1}%
	{\mdfsetup{%
			frametitle={%
				\tikz[baseline=(current bounding box.east),outer sep=0pt]
				\node[anchor=east,rectangle,fill=red!20]
				{\strut Proof~\theprf};}}
	}%
	{\mdfsetup{%
			frametitle={%
				\tikz[baseline=(current bounding box.east),outer sep=0pt]
				\node[anchor=east,rectangle,fill=red!20]
				{\strut Proof~\theprf:~#1};}}%
	}%
	\mdfsetup{innertopmargin=10pt,linecolor=red!20,%
		linewidth=2pt,topline=true,%
		frametitleaboveskip=\dimexpr-\ht\strutbox\relax
	}
	\begin{mdframed}[]\relax%
		\label{#2}}{\qed\end{mdframed}}
%%%%%%%%%%%%%%%%%%%%%%%%%%%%%%

%Definition
\newcounter{defn}[section] \setcounter{defn}{0}
\renewcommand{\thedefn}{\arabic{chapter}.\arabic{section}.\arabic{defn}}
\newenvironment{defn}[2][]{%
	\refstepcounter{defn}%
	\ifstrempty{#1}%
	{\mdfsetup{%
			frametitle={%
				\tikz[baseline=(current bounding box.east),outer sep=0pt]
				\node[anchor=east,rectangle,fill=gray!20]
				{\strut Definition~\thedefn};}}
	}%
	{\mdfsetup{%
			frametitle={%
				\tikz[baseline=(current bounding box.east),outer sep=0pt]
				\node[anchor=east,rectangle,fill=gray!20]
				{\strut Definition~\thedefn:~#1};}}%
	}%
	\mdfsetup{innertopmargin=10pt,linecolor=gray!20,%
		linewidth=2pt,topline=true,%
		frametitleaboveskip=\dimexpr-\ht\strutbox\relax
	}
	\begin{mdframed}[nobreak=true]\relax%
		\label{#2}}{\end{mdframed}}
	
%Fancy Box
\newcounter{fancybox}[section] \setcounter{fancybox}{0}
\renewcommand{\thefancybox}{\arabic{chapter}.\arabic{section}.\arabic{fancybox}}
\newenvironment{fancybox}[2][]{%
	\refstepcounter{fancybox}%
	\ifstrempty{#1}%
	{\mdfsetup{%
			frametitle={%
				\tikz[baseline=(current bounding box.east),outer sep=0pt]
				\node[anchor=east,rectangle,fill=orange!20]
				{\strut ~\thefancybox};}}
	}%
	{\mdfsetup{%
			frametitle={%
				\tikz[baseline=(current bounding box.east),outer sep=0pt]
				\node[anchor=east,rectangle,fill=orange!20]
				{\strut ~\thefancybox:~#1};}}%
	}%
	\mdfsetup{innertopmargin=10pt,linecolor=orange!20,%
		linewidth=2pt,topline=true,%
		frametitleaboveskip=\dimexpr-\ht\strutbox\relax
	}
	\begin{mdframed}[]\relax%
		\label{#2}}{\end{mdframed}}
