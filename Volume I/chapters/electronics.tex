\chapter{Electronics}
\thispagestyle{fancy}

\keyword{Electronics} is a multidisciplinary field that revolves around the study and manipulation of electrons and their behavior in different materials and devices. This field has revolutionized the way we communicate, compute, and interact with the world, giving rise to an unprecedented era of technological advancements. Electronics encompasses a vast range of topics, from the fundamental principles of semiconductor physics to the design and development of intricate integrated circuits, enabling the creation of devices that have become an integral part of our daily lives. Key Concepts in Electronics include the following:

\begin{itemize}
	\item \keyword{Semiconductors}: Semiconductors are materials that have properties between conductors (like metals) and insulators. The behavior of electrons in semiconductors is crucial to the operation of electronic devices. By manipulating the conductivity of semiconductors, devices like transistors and diodes can be created.

    \item \keyword{Transistors}: Transistors are fundamental building blocks of modern electronics. They act as switches or amplifiers, controlling the flow of electrical current by varying a small input signal. Transistors are the core components of integrated circuits (ICs), which power computers, smartphones, and a myriad of other devices.

    \item \keyword{Integrated Circuits}: Integrated circuits (ICs) are miniaturized collections of electronic components, such as transistors, resistors, and capacitors, fabricated on a single semiconductor wafer. They range from simple logic gates to complex microprocessors, enabling the creation of highly compact and powerful devices.

    \item \keyword{Digital} vs. \keyword{Analog}: Electronics deals with both digital and analog signals. Digital electronics represent information using discrete values (0s and 1s), forming the basis of computers and digital communication systems. Analog electronics, on the other hand, work with continuous signals and are essential for tasks like audio processing.

    \item \keyword{Circuit Design}: Circuit design involves creating schematics and layouts for electronic circuits. Design considerations include selecting components, optimizing power consumption, managing heat dissipation, and ensuring reliable operation.

    \item \keyword{Power Electronics}: Power electronics focuses on the conversion and control of electrical power. This field is vital for technologies such as power supplies, motor drives, renewable energy systems, and electric vehicles.

    \item \keyword{Signal Processing}: Signal processing involves manipulating signals to extract, enhance, or encode information. It's used in applications like image and audio processing, telecommunications, and control systems.

    \item \keyword{Microcontrollers} and \keyword{Microprocessors}: Microcontrollers are small computers on a single chip, often used to control specific tasks in devices like household appliances and automotive systems. Microprocessors are the brains of computers and smartphones, executing complex instructions and tasks.

    \item \keyword{Printed Circuit Boards} (PCBs): PCBs provide a platform for connecting and mounting electronic components. They are found in virtually all electronic devices and play a crucial role in determining a device's functionality and reliability.
\end{itemize}


\section{Basic Definitions and Notations}

\begin{defn}[Conductors]{defn:conductor}
An electrical \keyword{conductor} is a substance where the electrons are mobile.
\end{defn}

Copper and Aluminum are excellent conductors and other metals such as iron and steel range from fair to good conductors. There is no such thing as a perfect conductor. A \keyword{superconductor} is a material that loses practically all of its resistance (but is still not perfect).

\begin{interestnote}
One of the best conductors at room temperature is pure elemental silver. 
\end{interestnote}

\begin{defn}[Insulators]{defn:insulator}
An electrical \keyword{insulator} is a substance that prevents current from flowing.
\end{defn}

Most gases are good electrical insulators. Glass, dry wood, paper, and plastics are examples. 

\begin{interestnote}
Pure water is a good insulator but the slightest impurity can cause it to conduct current.
\end{interestnote}

\begin{defn}[Resistors]{defn:resistors}
\keyword{Resistors} are electrical components made to allow for the control of current flow. This is typically done by changing the size, shape, or adding impurities to a substance. Electrical resistance is measured in units called \keyword{ohms}.
\end{defn}

Resistance is often references as per a unit length (ohms per foot, etc). resistance also converts electrical energy into heat, and therefore low resistance is generally desired in electrical systems. If there is a current flowing through a resistive material, there will always be a potential difference across the object. There is no such thing as a resistance free object or an object with a truly infinite resistance.

\begin{defn}[Semiconductor]{defn:semiconductor}
A \keyword{semiconductor} is a substance that has a conductivity between an insulator and that of most metal conductors.
\end{defn}

Semiconductors are typically made from adding impurities or changing the temperature conditions. Silicon is a notable semiconductor that is common in electronic circuits. Indium or antimony are often used to dope silicon, selenium or gallium to create semiconductors. A \keyword{hole} is a shortage of an electron, and the movement of holes is often used as reference rather than the movement of electrons. When the charge carriers are electrons, a semiconductor is N-type (because the electrons are negatively charged). When the majority of the charge carriers are holes, a semiconductor is called P-type (since the absence of the negative electron creates a positive charge).

\begin{defn}[Current]{defn:current}
\keyword{Current} can be defined as the movement of charge carriers in a substance. Current is measured in terms of the number of charge carriers passing a single point in a second in units of \keyword{coulombs}\footnote{A coulomb is a measure of charge and is equivalent to approximately $6.24 \times 10^{18}$ charge carriers (electrons or holes).} per second. Another name for this unit is an \keyword{Ampere}.
\end{defn}

\begin{defn}{Static Electricity}{defn:static electricity}
\keyword{Static electricity} is a stationary electric charge that is typically generated when two objects with different electrical charges come into contact and then separate, leaving one with an excess of electrons (making it negatively charged) and the other with a deficit of electrons (making it positively charged). Unlike current electricity, which flows through conductors in a continuous circuit, static electricity remains in one place until it is discharged.
\end{defn}

\begin{defn}{Electromotive Force}{defn:emg}
\keyword{Electromotive force} (EMF), is a concept in physics and electrical engineering that represents the electrical \keyword{potential difference} or voltage generated by a source such as a battery, generator, or a power supply. EMF is measured in \keyword{volts} (V) and represents the energy per unit charge provided by the source to move electric charges carriers through a circuit. 
\end{defn}

An EMF of one volt $V$, accross a resistance $R$ of one ohm, will cause a current $I$ of one ampere to flow. this is a classical relationship of electricity and referred to generally as \keyword{Ohm's Law},
\begin{align}
V = I \times R
\end{align}

\begin{defn}{Conductance}{defn:conductance}
\keyword{Conductance} is the reciprocal of resistance and is measured in \keyword{siemens} (1/ohms). 
\end{defn}

\begin{defn}{Power}{defn:power}
\keyword{Power} is the hear that results when a current flows through a resistance and is measured in \keyword{watts} (W), which represent electrical power.
\begin{align}
P = V I = I^2 R = \frac{V^2}{R}
\end{align}
\end{defn}

There are many different units used to measure power and energy which typically depends on the context of the problem. A common unit used to measure energy by a physicist is the joule. One \keyword{joule} is the equivalent of one watt of power, dissipated for one second of time. Another unit is the \keyword{erg}, which is one ten-millionth of a joule. Most oftenly used to indicate cooling or heating capacity of air-conditioning units is the \keyword{british thermal unit} (Btu), which is 1055 joules. The energy gained by a single electron in an electric field of 1 volt is $1.6 \times 10^{-19}$ joules, which is called an \keyword{electron volt} (eV). The work needed to raise one pound by a distance of one foot (not including any friction) is 1.356 joules, and is called the \keyword{foot-pound} (ft-lb).

\begin{defn}{Alternating Current}{defn:ac}
\keyword{Alternating Current} (AC) is an electric current that reverses direction at regular intervals. In AC, the flow of electric charge periodically changes direction, typically oscillating back and forth in a sinusoidal (wave-like) pattern. The voltage and current in an AC circuit continuously alternate between positive and negative values, completing one full cycle within a defined period, usually measured in hertz (Hz).
\end{defn}

In America, the frequency of utility electricity is 60 Hz (the current goes through a complete cycle every $\frac{1}{60}$ second. \keyword{Rectification} is a process in which AC is converted to DC (most commonly done using a \keyword{diode}.

\begin{defn}{Direct Current}{defn:dc}
\keyword{Direct Current} (DC) is an electrical current that flows continuously in one direction without changing polarity. In a DC circuit, electric charge carriers move from the negative terminal of a power source (like a battery or generator) to the positive terminal in a steady, unidirectional flow.
\end{defn}











