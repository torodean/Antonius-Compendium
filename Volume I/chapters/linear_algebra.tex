\chapter{Linear Algebra}
\thispagestyle{fancy}

\keyword{Linear algebra} is a branch of mathematics that focuses on the study of vector spaces, linear transformations, and systems of linear equations. It provides a powerful framework for representing and solving problems involving linear relationships between variables. Key concepts in linear algebra include:

\begin{itemize}
	\item \keyword{Vectors}: Vectors are mathematical objects that represent magnitude and direction. They can be represented as ordered lists of numbers and are used to describe quantities with both magnitude and direction, such as velocity and force.
	
	\item \keyword{Vector Spaces}: A vector space is a set of vectors that satisfy certain properties under addition and scalar multiplication. It forms the foundation of linear algebra and allows for the study of linear combinations and transformations.
	
	\item \keyword{Matrices}: Matrices are rectangular arrays of numbers or elements, organized into rows and columns. They are used to represent linear transformations and to solve systems of linear equations.
	
	\item \keyword{Linear Transformations}: Linear transformations are functions that preserve the structure of vector spaces. They map vectors from one vector space to another while maintaining properties like linearity and preservation of the origin.
	
	\item \keyword{Eigenvalues} and \keyword{Eigenvectors}: In linear algebra, eigenvalues and eigenvectors are associated with linear transformations. Eigenvectors are special vectors that remain in the same direction after a linear transformation, and eigenvalues represent how much the eigenvectors are scaled during the transformation.
\end{itemize}

Linear algebra finds applications in various fields, including physics, computer graphics, engineering, data science, and economics. It plays a fundamental role in solving systems of linear equations, understanding linear transformations, and providing tools to analyze complex systems with multiple variables and interactions. Moreover, it forms the basis for more advanced mathematical concepts and techniques in areas like optimization, machine learning, and numerical analysis.


