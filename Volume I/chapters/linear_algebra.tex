\chapter{Linear Algebra}
\thispagestyle{fancy}

\keyword{Linear algebra} is a branch of mathematics that focuses on the study of vector spaces, linear transformations, and systems of linear equations. It provides a powerful framework for representing and solving problems involving linear relationships between variables. Key concepts in linear algebra include:

\begin{itemize}
	\item \keyword{Vectors}: Vectors are mathematical objects that represent magnitude and direction. They can be represented as ordered lists of numbers and are used to describe quantities with both magnitude and direction, such as velocity and force.
	
	\item \keyword{Vector Spaces}: A vector space is a set of vectors that satisfy certain properties under addition and scalar multiplication. It forms the foundation of linear algebra and allows for the study of linear combinations and transformations.
	
	\item \keyword{Matrices}: Matrices are rectangular arrays of numbers or elements, organized into rows and columns. They are used to represent linear transformations and to solve systems of linear equations.
	
	\item \keyword{Linear Transformations}: Linear transformations are functions that preserve the structure of vector spaces. They map vectors from one vector space to another while maintaining properties like linearity and preservation of the origin.
	
	\item \keyword{Eigenvalues} and \keyword{Eigenvectors}: In linear algebra, eigenvalues and eigenvectors are associated with linear transformations. Eigenvectors are special vectors that remain in the same direction after a linear transformation, and eigenvalues represent how much the eigenvectors are scaled during the transformation.
\end{itemize}

Linear algebra finds applications in various fields, including physics, computer graphics, engineering, data science, and economics. It plays a fundamental role in solving systems of linear equations, understanding linear transformations, and providing tools to analyze complex systems with multiple variables and interactions. Moreover, it forms the basis for more advanced mathematical concepts and techniques in areas like optimization, machine learning, and numerical analysis.



\section{Linear Systems}

\begin{defn}[Linear Combination]{defn:linear combination}
	A \keyword{linear combination} of $x_0,..., x_n$ has the form 
	\begin{align*}
	a_0x_0 + a_1x_1 + a_2x_2 + \cdots + a_nx_n,
	\end{align*}
	where the numbers $a_0,..., a_n \in \R$ are the combination’s coefficients. A linear
	equation in the variables $x_0,..., x_n$ has the form $a_0x_0 + a_1x_1 + a_2x_2 + \cdots +
	a_nx_n = d$, where $d \in\R$ is the constant. An $n$-tuple $(s_0, s_1,..., s_n) \in\R^n$ is a solution of, or satisfies, that equation
	if substituting the numbers $s_0,..., s_n$ for the variables gives a true statement: $a_0s_0 + a_1s_1 + · · · + a_ns_n = d$. A system of linear equations
	\begin{align*}
	a_{0,0}x_0 + a_{0,1}x_1 + ... + a_{0,n}x_n &= d_0 \\
	a_{1,0}x_0 + a_{1,1}x_1 + ... + a_{1,n}x_n &= d_1 \\
	a_{2,0}x_0 + a_{2,1}x_1 + ... + a_{2,n}x_n &= d_2 \\
	&\vdots \\
	a_{m,0}x_0 + a_{m,1}x_1 + ... + a_{2,n}x_n &= d_m
	\end{align*}
	has the solution $(s_0, s_1,..., s_n)$ if that n-tuple is a solution of all of the equations.
\end{defn}

\begin{theo}[Gauss's Method]{theo:gauss's method}
	If a linear system is changed to another by one of the following operations, then the two systems have the same set of solutions.
	\begin{enumerate}
		\item An equation is swapped with another.
		\item An equation has both sides multiplied by a non-zero constant.
		\item An equation is replaced by the sum of itself and a multiple of another.
	\end{enumerate}
\end{theo}
\begin{proof}[Proof for theorem \ref{theo:gauss's method} item 1]
	The proof for this one is trivial. If we have a system of linear equations with a tuple that satisfies those equations, then by definition \ref{defn:linear combination}, all the equations hold true with that tuple. By swapping any of them, they will still hold true, and thus swapping them has no effect on the set of solutions.
\end{proof}
\begin{proof}[Proof for theorem \ref{theo:gauss's method} item 2]
	The proof for this one is also trivial. Consider a system of linear equations with a tuple $(s_0, s_1,..., s_n) \in\R^n$ that satisfies those solutions. Then by definition \ref{defn:linear combination}, for all $i\in\N$, we have a true statement $a_{i,0}x_0 + a_{i,1}x_1 + ... + a_{i,n}x_n = d_i$. Suppose we multiply both sides of this equation by some $p\in\R$, where $p \neq 0$. Then we have $p(a_{i,0}s_0 + a_{i,1}s_1 + ... + a_{i,n}s_n)= pd_i$. Since the tuple is a solution to the equation, there has to exist some $pd_i$ value to satisfies this equation call it $d'_i$ (the prime representing our new linear system), giving us $a'_{i,0}s_0 + a'_{i,1}s_1 + ... + a'_{i,n}s_n) = pd_i = d'_i$, where $a'_i$ are some new coefficients $pa_i$ that satisfy the equation and $d'_i$ is some new constant. Our tuple will satisfy this equation, and since none of the other equations changed, it will also satisfy them. Therefore, our solution satisfies this new linear system. Since the solution was arbitrary, this implies that all solutions are also satisfied.
\end{proof}
\begin{proof}[Proof for theorem \ref{theo:gauss's method} item 3]
	The proof for this one is also trivial. Consider a system of linear equations with a tuple $(s_0, s_1,..., s_n) \in\R^n$ that satisfies those solutions. Let $p \in \R$. Take two equations in the system $a_{i,0}x_0 + a_{i,1}x_1 + ... + a_{i,n}x_n = d_i$ and $a_{j,0}x_0 + a_{j,1}x_1 + ... + a_{j,n}x_n = d_j$. We can construct a new equation by adding the first to a multiple of the second, say $d_i + pd_j$. this gives $d_i + pd_j = a_{i,0}x_0 + a_{i,1}x_1 + ... + a_{i,n}x_n + p(a_{j,0}x_0 + a_{j,1}x_1 + ... + a_{j,n}x_n)$. Since our tuple is a solution to both of these equations, we know that the first term $d_i$ and the second term $pd_j$ (when our tuple is plugged into the variables) will satisfy the new expression. Thus, since this tuple is a solution to this equation, and since none of the other equations changed, it will also satisfy them. Therefore, our solution satisfies this new linear system. Since the solution was arbitrary, this implies that all solutions are also satisfied.
\end{proof}

\begin{defn}[Gaussian Operations]{defn:gaussian operations}
	The three operations from theorem \ref{theo:gauss's method} are called \keyword{Gaussian operations} (or elementary reduction operations, or row operations). They are swapping, multiplying by a scalar, and row combination.
\end{defn}


\begin{defn}[Echelon Form]{defn:echelon form}
	In a system of equations, the initial variable with a non-zero coefficient within each row is referred to as the leading variable for that row. A system is said to be in \keyword{echelon form} when every leading variable, except for the first row's leading variable, is positioned to the right of the leading variable in the row above it. Additionally, any rows containing coefficients of all zeros are situated at the bottom. 
\end{defn}


In an echelon form linear system the variables that are not leading are \keyword{free}. An example of echelon form follows. Consider the equations $x - 4y + 2z = -1$, $7z + 3y = 10$, and $ z = 10$. The echelon form of these is 
\begin{align}
1x - 4y + 2z &= -1 \\
0x+3y + 7z &= 10 \\
0x+0y+1z &= 2
\end{align}
I added coefficients to each term in order to better demonstrate how the variables `line up.' This form is analogous to representing data in matrix format. For these equations, a matrix representation would be written as

\begin{align}
\begin{bmatrix}
1 & -4 & 2 \\
0 & 3 & 7 \\
0 & 0 & 1
\end{bmatrix}
\begin{bmatrix}
x \\
y \\
z
\end{bmatrix}
=
\begin{bmatrix}
-1 \\
10 \\
2
\end{bmatrix} 
\hspace{1cm} \text{or} \hspace{1cm} 
\left(
\begin{array}{ccc|c}
1 & -4 & 2 & -1 \\
0 & 3 & 7 & 10\\
3 & 0 & 1 & 2
\end{array}
\right)
\end{align}



\begin{defn}[Matrix]{defn:matrix}
	An $m\times n$ matrix is a rectangular array of numbers with $m$ rows and $n$ columns (2-dimensional). Each number in the matrix is an entry which has an index in each dimension. A matrix is typically denoted with an upper case letter. A matrix is typically a 2-dimensional object but can be mathematically extended into higher dimensions. for example, a $3x5$ matrix with elements $a_{mn}$ is denoted
	\begin{align*}
	A = 
	\begin{bmatrix}
	a_{11} & a_{12} & a_{13} & a_{14} & a_{15} \\
	a_{21} & a_{22} & a_{23} & a_{24} & a_{25} \\
	a_{31} & a_{32} & a_{33} & a_{34} & a_{35} \\
	\end{bmatrix}
	\end{align*}
\end{defn}

\begin{defn}[Homogeneous Solutions]{defn:homogeneous solution}
	A linear equation is \keyword{homogeneous} if it has a constant of zero and thus can be written
	\begin{align}
	a_0x_0+a_1x_1+\cdots+a_nx_n=0
	\end{align}
\end{defn}


\begin{lemma}[Homogeneous solution]{lemma:homogeneous linear equation solution}
	Any homogeneous linear system has a solution set of the form
	\begin{align*}
	\{c_1\vec{\alpha}_1+\cdots++c_n\vec{\alpha}_n| c_1,...c_n \in \R\},
	\end{align*}
	where $n$ is the number of free variables in an echelon form of the system.
\end{lemma}


\begin{theo}[General Solution: Particular + Homogeneous]{theo:general linear equation solution}
	Any linear system has a solution set of the form
	\begin{align*}
	\{\vec{p}+c_1\vec{\alpha}_1+\cdots++c_n\vec{\alpha}_n| c_1,...c_n \in \R\},
	\end{align*}
	where $\vec{p}$ is a particular solution and $n$ is the number of free variables that the system has after a Gaussian reduction.
\end{theo}
