\chapter{Quantum Mechanics}
\thispagestyle{fancy}

\keyword{Quantum mechanics} is a fundamental branch of physics that describes the behavior of matter and energy at the smallest scales, such as subatomic particles and photons. It provides a unique and revolutionary framework for understanding the peculiar and counterintuitive behavior of particles at the quantum level. Key concepts and principles of quantum mechanics include:

\begin{itemize}
	\item \keyword{Wave-Particle Duality}: One of the central tenets of quantum mechanics is the wave-particle duality. It states that particles, such as electrons and photons, exhibit both particle-like and wave-like characteristics. They can be described by wave functions, which represent probabilities of finding a particle at different locations.

	\item \keyword{Quantization of Energy}: Quantum mechanics introduced the concept of quantized energy levels, where energy levels of particles are restricted to discrete values rather than continuous values. This is exemplified in the energy levels of electrons in an atom, resulting in the discrete emission and absorption of photons.

	\item \keyword{Uncertainty Principle}: The Heisenberg uncertainty principle states that it is impossible to simultaneously know both the position and momentum of a particle with absolute precision. The more accurately one quantity is known, the less precisely the other can be determined. This fundamental limitation is inherent in quantum mechanics.

	\item \keyword{Quantum Superposition}: Quantum systems can exist in a state of superposition, where they are in multiple states simultaneously. For example, an electron can exist in a superposition of spin-up and spin-down states until measured, at which point it collapses into one definite state.

	\item \keyword{Quantum Entanglement}: Quantum entanglement is a phenomenon where the properties of two or more particles become correlated in such a way that the state of one particle is directly related to the state of another, regardless of distance. This has profound implications for quantum information and potential applications in quantum computing.

	Quantum Mechanics and Measurement: The process of measurement in quantum mechanics is non-deterministic. Upon measurement, the system's wave function collapses to a specific state corresponding to the observed measurement outcome, introducing inherent randomness into quantum events.

\end{itemize}
Quantum mechanics has revolutionized our understanding of the subatomic world and is the foundation for modern technologies such as transistors, lasers, and MRI machines. While it has proven to be highly successful in describing the behavior of particles on a small scale, it also challenges our classical intuition and raises profound philosophical questions about the nature of reality and our perception of the universe.