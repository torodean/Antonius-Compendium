\chapter{Quantum Mechanics}
\thispagestyle{fancy}

\keyword{Quantum mechanics} is a fundamental branch of physics that describes the behavior of matter and energy at the smallest scales, such as subatomic particles and photons. It provides a unique and revolutionary framework for understanding the peculiar and counterintuitive behavior of particles at the quantum level. Key concepts and principles of quantum mechanics include:

\begin{itemize}
	\item \keyword{Wave-Particle Duality}: One of the central tenets of quantum mechanics is the wave-particle duality. It states that particles, such as electrons and photons, exhibit both particle-like and wave-like characteristics. They can be described by wave functions, which represent probabilities of finding a particle at different locations.

	\item \keyword{Quantization of Energy}: Quantum mechanics introduced the concept of quantized energy levels, where energy levels of particles are restricted to discrete values rather than continuous values. This is exemplified in the energy levels of electrons in an atom, resulting in the discrete emission and absorption of photons.

	\item \keyword{Uncertainty Principle}: The Heisenberg uncertainty principle states that it is impossible to simultaneously know both the position and momentum of a particle with absolute precision. The more accurately one quantity is known, the less precisely the other can be determined. This fundamental limitation is inherent in quantum mechanics.

	\item \keyword{Quantum Superposition}: Quantum systems can exist in a state of superposition, where they are in multiple states simultaneously. For example, an electron can exist in a superposition of spin-up and spin-down states until measured, at which point it collapses into one definite state.

	\item \keyword{Quantum Entanglement}: Quantum entanglement is a phenomenon where the properties of two or more particles become correlated in such a way that the state of one particle is directly related to the state of another, regardless of distance. This has profound implications for quantum information and potential applications in quantum computing.

	Quantum Mechanics and Measurement: The process of measurement in quantum mechanics is non-deterministic. Upon measurement, the system's wave function collapses to a specific state corresponding to the observed measurement outcome, introducing inherent randomness into quantum events.

\end{itemize}
Quantum mechanics has revolutionized our understanding of the subatomic world and is the foundation for modern technologies such as transistors, lasers, and MRI machines. While it has proven to be highly successful in describing the behavior of particles on a small scale, it also challenges our classical intuition and raises profound philosophical questions about the nature of reality and our perception of the universe.





\section{Introduction}

The quantum theory first arose from studies of radiation. Many historical developements lead to the formulation of quantum mechanics, which include \keyword{Planck's radiation law}, the \keyword{Einstein-Debye theory of specific heats}, the \keyword{Bohr atom}, \keyword{de Broglie's matter waves}, and others. In the years leading up to 1920, the picture of space as 3-dimensional with time being separate was changed by Einstein. Following this, the rules for motions of particles was found to be incorrect. The mechanical rules for inertia and forces established by Newton was found to be wrong\footnote{Or at least only accurate as an approximation in certain cases.}. Around 1920, \keyword{quantum mechanics} was discovered which explained how things on small scale behave nothing like things on the larger scales. According to quantum mechanics, the things on a small scale behave so unnaturally, that it is hard to understand outside of an analytic and mathematical framework. \keyword{Quantum mechanics} is the description of the behavior of matter and light in all its details and, in particular, of the happenings on an atomic scale. 

When thinking of atoms from the atomic hypothesis (outlined in section \ref{section:The Atomic Hypothesis}), we can think of some mysterious questions and paradoxes that seem to arise.

\begin{questions}
	\item If the atoms are made out of plus and minus charges, why don’t the minus charges simply sit on top of the plus charges (they attract each other) and get so close as to completely cancel them out? 
	\item Why are atoms so big? 
	\item Why is the nucleus at the center with the electrons around it?
\end{questions}

An atom has a diameter on the scale of about $10^{-8}$ cm. Within the atom, the nucleus only has a diameter on the scale of $10^{-13}$, which is much smaller! The Heisenberg uncertainty principal $\Delta x \Delta p \geq \hbar/2$ was a rule proposed by Werner Karl Heisenberg to answer these questions above. Take an electron for example. Why is it that the electron does not fall into the nucleus of an atom? The uncertainty principal states that if we know the position of the particle, the momentum would have to be very large - the electron would have a very high kinetic energy. This energy would cause the electron to break away from the nucleus. This is also why there is still a little `jiggle' in the atoms at absolute zero. They need to keep moving in order to follow this rule.

\begin{questions}
	\item How does the nucleus fit into this rule?
	\item Do the particles within the nucleus that are so closely attracted obey this rule? Perhaps this is why there is so much energy within a nucleus.
	\item Where does the energy come from in these systems that keep matter moving?
\end{questions}

Another interesting change that quantum physics brought about to the philosophy of physics is that \textit{it is not possible to predict exactly what will happen in any circumstance.} This is the statistical nature of quantum mechanics. According to quantum physics, it is fundamentally impossible to make a precise (\textit{exactly what will happen}) prediction of nature. One of the consequences of quantum mechanics is that things which we used to consider as waves also behave like particles, and particles behave like waves; in fact everything behaves the same way. There is no distinction between a wave and a particle. Quantum mechanics therefore unifies the idea of the field and its waves with the idea of particles into one. When the frequency is low, the field aspect of a phenomenon is more evident. As frequencies increase, the particle aspects of a phenomenon become more evident within equipment used in experimentation. The \keyword{photon}\footnote{A photon in some contexts can be thought of as just a `packet of light'.} is a new particle introduced alongside the electron, neutron, and proton. The new view of the interaction of electrons and photons that is electromagnetic theory, but with everything quantum-mechanically correct, is called \keyword{quantum electrodynamics}.


\begin{questions}
	\item Is there a new mathematical framework that can act like both particle and waves?
 	\item From my previous understanding (perhaps I'm completely wrong), treating a system with a wave-function is only useful until the wave function collapses (a measurement is taken). After such, there is no reversible process to return (mathematically) to the wave function from the collapsed state. Perhaps if another dimension (or a few) was added, the wave function could be `rotated' into a different dimension, which would demonstrate the collapse, and then rotated back preserving the original wave function. This would illiminate any discontinuities.
\end{questions}




\subsection{The Stern-Gerlach Experiment}

Begin with silver (Ag) atoms heated in an oven. The oven has a small hole where some of the silver atoms can escape, creating a beam of silver atoms. The beam goes through a collimator and is then subjected to an inhomogeneous magnetic field created by a oaur of pole pieces (one of which has a very sharp edge). A sikver aton has 47 electrons, where 46 form a spherically symmetrical electron cloud with no angular momentum. The silver atom as a whole, has an intrinsic (as opposed to orbital) angular momentum due solely to the spin (ignoring the nuclear spin) angular momentum of the single 47th (5s) electron. The nucleus of the atom is $\approx 2 \times 10^5$ times heavier than the electron. Therefore, the magnetic moment $\vec{\mu}$ of the atom equal to the spin magnetic moment of the 47th electron, and thus proportional to the electron spin $\vec{S}$, giving $\vec{\mu} \propto \vec{S}$, where the proportionality factor is $\frac{e}{m_ec}$ to an accuracy of about 0.2\%\cite{bib:Modern Quantum Mechanics J.J Sakurai}.

The interaction energy of the magnetic moment with the magnetic field is $-\vec{\mu}\cdot\vec{B}$ and the $z$-component of the force experienced by the atom is 
\begin{align}
F_z = \frac{\partial}{\partial z}(\vec{\mu}\cdot \vec{B}) \simeq \mu_z \frac{\partial B_z}{\partial z}.
\end{align}
When $\mu_z > 0$ ($S_z < 0$), atoms will experience an downward force. Similarly, when $\mu_z < 0$ ($S_z > 0$), atoms will experience an upward force. The beam is then expected to be split according to the values of $\mu_z$. In the sense of a classical spinning object, this setup would expect all values of $\mu_z$ to be possible between $|\vec{\mu}|$ and $-|\vec{\mu}|$, since the atoms in the oven are randomly oriented. What was experimentally observed, however, was that there was only two `spots' observed, corresponding to one `up' ($S_z^+$) and one `down'($S_z^-$) orientation, splitting the original silver beam into two distinct components. These components are multiples of the fundamental units of angular momentum which are $S_z=\frac{\hbar}{2}$ and $S_z=-\frac{\hbar}{2}$.

\begin{questions}
	\item There are 2 stable isotopes of silver (AG-107 and Ag-109). They have a molar ratio of about .52 to .48 \cite{bib:IUPAC Periodic Table, bib:Table Of Nuclides}. Could this somehow explain the descrepency in this experiment? 
\end{questions}

\subsubsection{Sequential Stern-Gerlach Experiments}

Consider a sequential Stern-Gerlach experiment where the silver beam passes through multiple Stern-Gerlach apparatuses (SGA) in sequence. The first setup is by blocking the $S_z^-$ component from the first SGA and allow the remaining $S_z^+$ component to enter a second SGA. In this case, only the $S_z^+$ component leaves the second SGA. Another setup follows where the first SGA remains the same, while allowing only the $S_z^+$ through, but the second SGA now has an inhomogeneous magnetic field in the $x$ direction rather than the $z$ direction. In this case, the $S_z^+$ beam entering the second SGA is split into two components, $S_x^+$ and $S_x^-$. Yet another setup follows where we use three SGAs. The first SGA will be setup as the others, where the silver bean is split into $S_z^+$ and $S_z^-$. The second SGA takes only the $S_z^+$ beam using an inhomogeneous magnetic field in the $x$ direction and splits the beam into a $S_x^+$ and $S_x^-$ beam. The third SGA takes only the $S_x^+$ beam and uses a magnetic field in the $z$ direction (the same as the first SGA). In this case, the final output beam of the third SGA has both $S_z^+$ and $S_z^-$ components.

This example of sequential Stern-Gerlach experiments is often used to illustrate that in quantum mechanics, you cannot determine both $S_z$ and $S_x$ simultaneously. This is typically understood in that any measurement in one apparatus negates any previously obtained information measured. This is a well regarded result and well accepted to not be due to the incompetence of experimentalists, but rather an inherent microscopic phenomena of matter itself. 

In classical mechanics, where angular momentum is defined as $\vec{L} = I\vec{\omega}$, there is no difficulty in determining the various components of $\omega_x$, $\omega_y$, and $\omega_z$ simultaneously by observing how fasts in object spins and in what direction. This effect is inherent to matter on a small scale and is a quantum mechanical property.









\section{Conventions and Notation}

The double-slit experiment is a fundamental experiment in quantum physics that demonstrates the wave-particle duality of particles, such as electrons and photons. It involves a barrier with two narrow slits through which particles can pass before hitting a screen. When particles are sent through the double slits one by one, they create an interference pattern on the screen, similar to the pattern created by waves. This suggests that particles exhibit wave-like behavior and interfere with themselves, producing areas of constructive and destructive interference. However, when the particles are observed or measured to determine which slit they pass through, the interference pattern disappears, and particles behave like individual particles, creating two distinct bands on the screen. This can be summarized using complex numbers and probabilities.

A probability density function $P$ describing an event in an ideal experiment is given by the square of the absolute value of the probability amplitude $|A|^2$, where $A$ is a complex number. When an event can occur in several alternate ways, the probability amplitude for hte event is the sum of the probability amplitudes for each way considered separately, giving interference. In mathematical terms this gives $A=A_1+A_2$ with $P=|A_1+A_2|^2$. When conducting an experiment capable of discerning between different alternatives, the probability of an event occurring is the sum of the probabilities associated with each alternative. Consequently, any interference or overlap between the alternatives is eliminated and you get $P=P_1+P_2$.

\begin{defn}[Expectation Value \label{Expectation Value Definition}]{1}
	If $P(x)$ denotes a probability density function of $x$, then the \keyword{expectation Value} (or average) of $x$, denoted $\braket{x}$ as
	\begin{align}
		\braket{x} = \int_{-\infty}^{\infty}xP(x) dx \label{expectation Value equation}
	\end{align}
	For discrete values of $x_i$ each with a correlated probability $P_i$, then the expectation value is 
	\begin{align}
		\braket{x} = \sum_{i}x_iP_i \label{expectation Value equation discrete}
	\end{align}
\end{defn}
 
A physical state is represented by a \keyword{state vector} in a complex vector space called a \keyword{ket}. This state is postulated to contain the complete information about the physical state of a system. An \keyword{observable} (momentum, spin components, etc) is represented by an \keyword{operator} within the vector space. The operators act on a ket from the left. In quantum physics, operators play a fundamental role in describing the behavior of quantum systems and the mathematical representation of physical observables. Quantum operators are mathematical entities that represent physical properties, transformations, and measurements within the framework of quantum mechanics. They are used to describe the behavior of quantum states, interactions, and measurements in a mathematical and abstract way. For example, consider the Hamiltonian operator $\hat{\text{\textbf{H}}}$ acted on a ket $\ket{x}$, represented by $\hat{\text{\textbf{H}}}\ket{x}$. This is not the operator `times' the ket, but rather the operator acting on the ket. In this example, the hamiltonian operator in classical mechanics is defined by $\hat{\text{\textbf{H}}} = \frac{\hbar}{2m}\nabla^2+V(\vec{r,t})$. This would act on $\ket{x}$ by taking the partial differentials of $\ket{x}$ as found in the first term of $\hat{\text{\textbf{H}}}$.












