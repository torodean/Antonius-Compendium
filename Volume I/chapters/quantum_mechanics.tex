\chapter{Quantum Mechanics}
\thispagestyle{fancy}

\keyword{Quantum mechanics} is a fundamental branch of physics that describes the behavior of matter and energy at the smallest scales, such as subatomic particles and photons. It provides a unique and revolutionary framework for understanding the peculiar and counterintuitive behavior of particles at the quantum level. Key concepts and principles of quantum mechanics include:

\begin{itemize}
	\item \keyword{Wave-Particle Duality}: One of the central tenets of quantum mechanics is the wave-particle duality. It states that particles, such as electrons and photons, exhibit both particle-like and wave-like characteristics. They can be described by wave functions, which represent probabilities of finding a particle at different locations.

	\item \keyword{Quantization of Energy}: Quantum mechanics introduced the concept of quantized energy levels, where energy levels of particles are restricted to discrete values rather than continuous values. This is exemplified in the energy levels of electrons in an atom, resulting in the discrete emission and absorption of photons.

	\item \keyword{Uncertainty Principle}: The Heisenberg uncertainty principle states that it is impossible to simultaneously know both the position and momentum of a particle with absolute precision. The more accurately one quantity is known, the less precisely the other can be determined. This fundamental limitation is inherent in quantum mechanics.

	\item \keyword{Quantum Superposition}: Quantum systems can exist in a state of superposition, where they are in multiple states simultaneously. For example, an electron can exist in a superposition of spin-up and spin-down states until measured, at which point it collapses into one definite state.

	\item \keyword{Quantum Entanglement}: Quantum entanglement is a phenomenon where the properties of two or more particles become correlated in such a way that the state of one particle is directly related to the state of another, regardless of distance. This has profound implications for quantum information and potential applications in quantum computing.

	Quantum Mechanics and Measurement: The process of measurement in quantum mechanics is non-deterministic. Upon measurement, the system's wave function collapses to a specific state corresponding to the observed measurement outcome, introducing inherent randomness into quantum events.

\end{itemize}
Quantum mechanics has revolutionized our understanding of the subatomic world and is the foundation for modern technologies such as transistors, lasers, and MRI machines. While it has proven to be highly successful in describing the behavior of particles on a small scale, it also challenges our classical intuition and raises profound philosophical questions about the nature of reality and our perception of the universe.





\section{Introduction}

In the years leading up to 1920, the picture of space as 3-dimensional with time being separate was changed by Einstein. Following this, the rules for motions of particles was found to be incorrect. The mechanical rules for intertia and forces established by Newton was found to be wrong\footnote{Or at least only accurate as an approximation in certain cases.}. Around 1920, \keyword{quantum mechanics} was discovered which explained how things on small scale behave nothing like things on the larger scales. According to quantum mechanics, the things on a small scale behave so unnaturally, that it is hard to understand outside of an analytic and mathematical framework.

When thinking of atoms from the atomic hypothesis (outlined in section \ref{section:The Atomic Hypothesis}), we can think of some mysterious questions and paradoxes that seem to arise.

\begin{questions}
	\item If the atoms are made out of plus and minus charges, why don’t the minus charges simply sit on top of the plus charges (they attract each other) and get so close as to completely cancel them out? 
	\item Why are atoms so big? 
	\item Why is the nucleus at the center with the electrons around it?
\end{questions}

An atom has a diameter on the scale of about $10^{-8}$ cm. Within the atom, the nucleus only has a diameter on the scale of $10^{-13}$, which is much smaller! The Heisenberg uncertainty principal $\Delta x \Delta p \geq \hbar/2$ was a rule proposed by Werner Karl Heisenberg to answer these questions above. Take an electron for example. Why is it that the electron does not fall into the nucleus of an atom? The uncertainty principal states that if we know the position of the particle, the momentum would have to be very large - the electron would have a very high kinetic energy. This energy would cause the electron to break away from the nucleus. This is also why there is still a little `jiggle' in the atoms at absolute zero. They need to keep moving in order to follow this rule.

\begin{questions}
	\item How does the nucleus fit into this rule?
	\item Do the particles within the nucleus that are so closely attracted obey this rule? Perhaps this is why there is so much energy within a nucleus.
	\item Where does the energy come from in these systems that keep matter moving?
\end{questions}

Another interesting change that quantum physics brought about to the philosophy of physics is that \textit{it is not possible to predict exactly what will happen in any circumstance.} This is the statistical nature of quantum mechanics. According to quantum physics, it is fundamentally impossible to make a precise (\textit{exactly what will happen}) prediction of nature. One of the consequences of quantum mechanics is that things which we used to consider as waves also behave like particles, and particles behave like waves; in fact everything behaves the same way. There is no distinction between a wave and a particle. Quantum mechanics therefore unifies the idea of the field and its waves with the idea of particles into one. When the frequency is low, the field aspect of a phenomenon is more evident. As frequencies increase, the particle aspects of a phenomenon become more evident within equipment used in experimentation. The \keyword{photon}\footnote{A photon in some contexts can be thought of as just a `packet of light'.} is a new particle introduced alongside the electron, neutron, and proton. The new view of the interaction of electrons and photons that is electromagnetic theory, but with everything quantum-mechanically correct, is called \keyword{quantum electrodynamics}.
