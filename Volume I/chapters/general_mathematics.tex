\chapter{General Mathematics}
\thispagestyle{fancy}

Mathematics is a systematic field of study that deals with numbers, quantities, shapes, and patterns. It is often regarded as the language of science, as it provides a framework for analyzing and understanding various phenomena in the natural and social world. Mathematics involves a wide range of topics, including arithmetic, algebra, geometry, calculus, statistics, and more. It is based on rigorous logical reasoning and uses symbols and formulas to represent relationships and solve problems. Mathematics plays a crucial role in various fields, such as physics, engineering, economics, computer science, and many other disciplines, making it an essential tool for advancing knowledge and technology.


\section{Coordinate Systems}

\subsection{Cartesian Coordinates}

\begin{figure}
	\centering
	\begin{tikzpicture}[scale=1.5]
		% 2D Axis
		\draw[->] (-1,0) -- (1.5,0) node[right] {$x$};
		\draw[->] (0,-1) -- (0,1.5) node[above] {$y$};
		% 2D Grid
		\foreach \x in {1}
		\draw[dotted] (\x,0) -- (\x,\x);
		\foreach \y in {1}
		\draw[dotted] (0,\y) -- (\y,\y);
		% 2D Point
		\fill (1,1) circle (1pt) node[above right] {$(x, y)$};
		
		\begin{scope}[xshift=5cm]
			% 3D Axis
			\draw[->] (0,0,0) -- (1.5,0,0) node[right] {$x$};
			\draw[->] (0,0,0) -- (0,1.5,0) node[above] {$y$};
			\draw[->] (0,0,0) -- (0,0,2) node[below left] {$z$};
			% 3D Grid
			\foreach \x in {0}
			\draw[dotted] (\x,0,0) -- (\x,\x,0) -- (\x,\x,\x);
			\foreach \y in {0}
			\draw[dotted] (0,\y,0) -- (\y,\y,0) -- (\y,\y,\y);
			\foreach \z in {1}
			\draw[dotted] (0,0,1) -- (0,1,1) -- (1,1,1);
			\draw[dotted] (0,1,0) -- (0,1,1);
			% 3D Point
			\fill (1,1,1) circle (1pt) node[above right] {$(x, y, z)$};
		\end{scope}
	\end{tikzpicture}
	\caption{2-D and 3-D Cartesian Coordinate Systems with a single point located at one unit in each axes. \label{fig:Cartesian Coordinates}}
\end{figure}

\keyword{Cartesian coordinates}, also known as \keyword{rectangular coordinates}, are a system of representing points in a two-dimensional or three-dimensional space using a set of perpendicular axes (See figure \ref{fig:Cartesian Coordinates}). Cartesian coordinates can be mathematically extended into higher dimensions. 

In two-dimensional Cartesian coordinates, a point is located by specifying its distance (magnitude) along the x-axis and the y-axis from the origin (0,0), where the x-axis is horizontal, and the y-axis is vertical. The coordinates of a point are written as (x, y), where x represents the horizontal distance and y represents the vertical distance. In three-dimensional Cartesian coordinates, a point is located by specifying its distance along the x-axis, y-axis, and z-axis from the origin (0, 0, 0). The coordinates of a point are written as (x, y, z), where x, y, and z represent the distances along the three axes. In higher $n$-dimensions, a similar formulation is used where a point is located by specifying its distance along $n$-axes (typically denoted by the unit vector $\hat{e}_i$) from the origin $\vec{0}$. The coordinates are $(x_0,x_1,...,x_n)$, where $x_i$ represents the distance along the respective $n$ axes.


\subsection{Polar Coordinates}

\keyword{Polar coordinates} are a two-dimensional coordinate system  (see figure \ref{fig:polar spherical coords})used to describe points in a plane. Unlike the traditional Cartesian coordinates $(x, y)$, polar coordinates represent points using a distance ($r$) from the origin and an angle ($\theta$) measured counterclockwise from a reference line (usually the positive $x$-axis). The distance $r$ is always a non-negative value, representing the radial distance from the origin to the point. The polar coordinate system is especially useful when dealing with circular or radial patterns, as it simplifies calculations involving angles and distances. The conversion between polar coordinates and Cartesian coordinates can be done using trigonometric functions.

\begin{align}
	x = r\cos(\theta) &\hspace{2cm} y = r\sin(\theta) \\
	 r = \sqrt{x^2 + y^2} &\hspace{2cm} \theta = \arctan\left(\frac{y}{x}\right).
\end{align}

\begin{figure}[htbp]
	\centering
	\begin{tikzpicture}
		% Draw Cartesian axes
		\draw[->] (-1,0) -- (3,0) node[below right] {$x$};
		\draw[->] (0,-1) -- (0,3) node[above left] {$y$};
		\draw[dotted] (0,1.5) -- (2,1.5);
		\draw[dotted] (2,0) -- (2,1.5);
		
		% Draw point in Cartesian coordinates (x,y)
		\filldraw[blue] (2, 1.5) circle (2pt) node[above right] {$(x, y)$};
		
		% Draw radial line and angle in polar coordinates (r, theta)
		\draw[dashed] (0,0) -- (2, 1.5) node[midway, above left] {$r$};
		\draw (0.8,0) arc (0:33.69:0.8) node[midway, right] {$\theta$};

		\begin{scope}[xshift=7cm]
		% 3D Axis
		\draw[->] (0,0,0) -- (3,0,0) node[right] {$x$};
		\draw[->] (0,0,0) -- (0,3,0) node[above] {$y$};
		\draw[->] (0,0,0) -- (0,0,3) node[below left] {$z$};
		
		% 3D Grid
		\draw[dotted] (0,0,2) -- (0,2,2) -- (2,2,2);
		\draw[dotted] (0,2,0) -- (0,2,2);
		\draw[dotted] (0,2,0) -- (0,2,2);
		\draw[dotted] (0,0,0) -- (2,0,2) -- (2,2,2);
		\draw[dotted] (2,0,0) -- (2,0,2) -- (0,0,2);
		
		% 3D Point
		\filldraw[blue] (2,2,2) circle (2pt) node[above right] {$(x, y, z)$};
		
		% Draw radial line and angles
		\draw[dashed] (0,0,0) -- (2,2,2);
		\draw[dashed] (0,0,0) -- (2,2,2) node[midway, above left] {$r$};
		\draw[-] (0,0,.8) arc (-90:-83:5.5) node[midway, below] {$\phi$};
		\draw[-] (1,0,1) arc (0:31:1.5) node[midway, above right] {$\theta$};
		\end{scope}
	\end{tikzpicture}
	
	\caption{LEFT: The point $(x, y)$ in Cartesian coordinates can be represented as $(r, \theta)$ in polar coordinates, where $r$ is the distance from the origin to the point and $\theta$ is the angle measured counterclockwise from the positive $x$-axis. RIGHT: The point $(x, y, z)$ in Cartesian coordinates can be represented as $(r, \theta, \phi)$ in a modified spherical coordinate system, where $r$ is the distance from the origin to the point, $\theta$ is the angle measured counterclockwise from the positive $x$-axis towards the $y$ direction and $\phi$ is the angle from the $z$-axis.}
	\label{fig:polar spherical coords}
\end{figure}





\subsection{Spherical Coordinates}

\keyword{Spherical coordinates} (see figure \ref{fig:polar spherical coords}) are a system of representing points in 3D space using three parameters: radial distance $r$, polar angle $\theta$, and azimuthal angle $\phi$. This system is particularly useful for describing points on a sphere or points in a 3D space relative to a fixed origin. The radial distance $r$ represents the distance from the origin to the point. It is a non-negative value and measures how far the point is from the origin. It is important to note that polar angle and azimuthal angle can often be represented in different ways depending on the coordinate convention. Typically, polar angle represents the angle between the positive z-axis and the line connecting the origin to the point projected onto the $xy$-plane. It varies from $0^\circ$ (on the positive z-axis) to $180^\circ$ (on the negative z-axis). This `polar' angle in essence would then connect both poles on a planet. I prefer to use the convention that matches 2-dimensional polar coordinates, as it removes some confusion when converting between the two - therefore $\theta$ is defined as the same angle as it would be in 2 dimensions and then $\phi$ is the angle of that 2-dimensional plane rotated about the $z$-axis. The conversion between spherical coordinates and Cartesian coordinates can be done using trigonometric functions. For the convention I prefer, and matching figure \ref{fig:polar spherical coords}, we have

\begin{align}
	x = r\cos(\theta)\sin(\phi), &\hspace{1cm} y = r\sin(\theta), \hspace{1cm} z = r\cos(\theta)\cos(\phi), \hspace{1cm}
	r = \sqrt{x^2 + y^2 + z^2}
\end{align}

These can be shown to be equivalent to the more commonly used spherical coordinates. To demonstrate this, I will denote the coordinates defined above with the subscript $_0$. For a more common convention (the physics one), imagine we modify our coordinates so that the $x_0$ axis is the $y$ axis, The $y_0$ axis is the $z$ axis, and the $z_0$ axis is the $x$ axis. Then we further define $\theta$ as the angle from the $z$ axis to our radial line, which is therefore $\theta_0 = \frac{\pi}{2}-\theta$ or  $\theta = \frac{\pi}{2}-\theta_0$. In this change, $r$ does not change as we are not changing our line or angles (just denoting them differently) giving us $r_0\equiv r$, and $\phi$ also stays the same giving $\phi_0 \equiv \phi$. Plugging these into our coordinates above give us

\begin{align}
	x_0 = r_0\cos(\theta_0)\sin(\phi_0) &\implies y = r\cos(\frac{\pi}{2}-\theta)\sin(\phi) = r\sin(\theta)\sin(\phi) \\ 
y_0 = r_0\sin(\theta_0) &\implies z = r\sin(\frac{\pi}{2}-\theta) = r\cos(\theta) \\
z_0 = r_0\cos(\theta_0)\cos(\phi_0) &\implies x =  r\cos(\frac{\pi}{2}-\theta)\cos(\phi) = r\sin(\theta)\cos(\phi)\\
r_0 = \sqrt{x_0^2 + y_0^2 + z_0^2} &\implies r = \sqrt{x^2 + y^2 + z^2}
\end{align}

This conversion gives us the more commonly used spherical coordinate convention

\begin{align}	 
	x = r\sin(\theta)\cos(\phi), &\hspace{1cm} 
	y = r\sin(\theta)\sin(\phi), &\hspace{1cm} 
	z = r\cos(\theta), &\hspace{1cm}
	r = \sqrt{x^2 + y^2 + z^2}
\end{align}

Therefore we can see they are equivalent and can be used interchangeably.








\subsection{Cylindrical Coordinates}










\subsection{Elliptic Cylindrical Coordinates}







\section{Conventions and basic Definitions}






\section{Summation}

\begin{defn}[Summation Notation \label{Summation Definition}]{1}
	A sum is the result of addition. The symbol $\sum$ is used to represent the addition of a series of values. Let $..., x_{a-1}, x_a, x_{a+1}, ... ,x_{b-1}, x_b, x_{b+1}, ...$ be a fixed series $\mathbb{S}$, where each term $x_i\in\mathbb{S}$, $b, a \in \mathbb{N}$ are index values. Then the sum of values in the series from $a$ to $b$ is given by
	\begin{align}
		\sum_{i=a}^{b} x_i = x_a +  x_{a+1} + ... + x_{b-1} + x_b
	\end{align}
	In the context of summing over all of the values of a series for some index $i$, it is often written in different ways which are typically equivalent.
	\begin{align}
		\sum_{i\in\mathbb{S}} x_i \equiv \sum_{i} x_i
	\end{align}
	If the number of terms in the summation is infinite then it is called an infinite series.
\end{defn}

\begin{defn}[Product Notation \label{Product Definition}]{1}
	A product is the result of multiplication. The symbol $\prod$ is used to represent the multiplication of a series of values. Let $..., x_{a-1}, x_a, x_{a+1}, ... ,x_{b-1}, x_b, x_{b+1}, ...$ be a fixed series $\mathbb{S}$, where each term $x_i\in\mathbb{S}$, $b, a \in \mathbb{N}$ are index values. Then the product of values in the series from $a$ to $b$ is given by
	\begin{align}
		\prod_{i=a}^{b} x_i = x_a \cdot  x_{a+1} \cdot ... \cdot x_{b-1} \cdot x_b
	\end{align}
	In the context of multiplying over all of the values of a series for some index $i$, it is often written in different ways which are typically equivalent.
	\begin{align}
		\prod_{i\in\mathbb{S}} x_i \equiv \prod_{i} x_i
	\end{align}
	If the number of terms in the summation is infinite then it is called an infinite series.
\end{defn}

%Consider a finite sum of values in a sequence $x$ from index $a$ to $b$. If we square this sum we would get
%
%\begin{align}
%	\bigg[\sum_{i=a}^{b} x_i\bigg]^2 &= (x_a +  x_{a+1} + ... + x_{b-1} + x_b)^2 \\
%	&= (x_a +  x_{a+1} + x_{a+2} + ... + x_{b-1} + x_b)(x_a +  x_{a+1} + x_{a+2} + ... + x_{b-1} + x_b) \\
%	&= \hspace{1cm} x_a^2 + 2x_ax_{a+1} + 2x_ax_{a+2} + \cdots + 2x_ax_b \nonumber\\
%	& \hspace{1cm}+ 2x_{a+1}x_{a} + x_{a+1}^2 + 2x_{a+1}x_{a+2} + \cdots + 2x_{a+1}x_b \nonumber\\
%	& \hspace{1cm}+ 2x_{a+2}x_{a} + 2x_{a+2}a_{a+1} + x_{a+2}^2 + \cdots + 2x_{a+2}x_b \nonumber\\
%	& \hspace{1.5cm} \vdots \nonumber\\
%	& \hspace{1cm}+ 2x_{b}x_{a} + 2x_{b}a_{a+1} + 2x_{b}x_{a+2} + \cdots + x_b^2	
%\end{align}
%
%From this, a pattern can be observed. There exists an $x_i^2$ term for each $i \in [a,b]$. There then exists a term $2x_ax_b$ for  all $a \neq b$. Therefore, we can write this as
%
%\begin{align}
%	\bigg[\sum_{i=a}^{b} x_i\bigg]^2 &= \sum_{i=a}^{b} x_i^2 + 2\sum_{i=a}^{b}\sum_{j=a}^{b} x_ix_j (1-\delta_{ij}),
%\end{align}
%where $\delta_{ij}$ is the Kronecker delta function.
%\begin{defn}[Kronecker Delta]
%	The \keyword{Kronecker delta} $\delta_{ij}$ is a discrete version of the delta function
%	\begin{align}
%		\delta_{ij} = \begin{cases}
%			1, & \text{if } i = j \\
%			0, & \text{if } i \neq j \\
%		\end{cases}  \implies 1-\delta_{ij} = \begin{cases}
%		0, & \text{if } i = j \\
%		1, & \text{if } i \neq j \\
%		\end{cases} \hspace{2cm}
%	\end{align}
%\end{defn}




















\subsection{Pascal's Triangle\label{section:pascals_triangle}}

\keyword{Pascal's Triangle} is a mathematical arrangement of numbers that holds several interesting properties and has applications in various fields. Pascal's Triangle is constructed by starting with a single "1" at the top. Each subsequent row is created by placing "1"s on both ends, and each inner value is the sum of the two values directly above it.
\begin{equation}
\renewcommand{\arraystretch}{1.5}
\begin{tabular}{cccccccccccc}
&      &      &      &      &      &  1   &      &      &      &      &      \\
&      &      &      &      &  1   &      &  1   &      &      &      &      \\
&      &      &      &  1   &      &  2   &      &  1   &      &      &      \\
&      &      &  1   &      &  3   &      &  3   &      &  1   &      &      \\
&      &  1   &      &  4   &      &  6   &      &  4   &      &  1   &      \\
&  1   &      &  5   &      &  10  &      &  10  &      &   5  &      &   1   \\
\end{tabular}\label{pascals_triangle}
\end{equation}
The numbers in Pascal's Triangle are also known as \keyword{binomial coefficients}. The entry in the $n$th row and $k$th column represents the coefficient of the term $x^k$ in the expansion of $(x + 1)^n$. The values in Pascal's Triangle also correspond to combinations (binomial coefficients) that count the number of ways to choose $k$ items from a set of $n$ items without considering the order. Pascal's Triangle is symmetric along its central vertical axis. Each row is a palindrome. The sum of the numbers in each row is a power of 2 (see theorem \ref{thm:Summing Over All binomial coefficients}). In number theory, Pascal's Triangle is involved in finding patterns and relationships in numbers, and it is often connected to the study of the binomial series and power series in calculus. Pascal's Triangle can be extended to include negative row indices and even more complex expansions.

\section{Binomial Coefficients}

\begin{defn}[Binomial Coefficient \label{Binomial Coefficient Definition}]{1}
	The binomial coefficient, ${{n}\choose{k}}$, often denoted as "n choose k," is a mathematical concept used in combinatorics to represent the number of ways to choose k items from a set of n distinct items, regardless of the arrangement of the chosen elements. 	
	\begin{align}
		{{n}\choose{k}}&=\frac{n!}{(n-k)!k!}
	\end{align}
\end{defn}

To begin with, a useful idea is to sum each term of the binomial coefficient which will be used later. First, by definition of the binomial coefficient, we can write
\begin{align}
	{{n}\choose{k}}&=\frac{n!}{(n-k)!k!} \label{binomial-coefficient-defn}\\
	&=\frac{(n-1)!n}{(n-k)!k!}\\&
	=\frac{(n-1)!n-k(n-1)!+k(n-1)!}{(n-k)!k!}\\
	&=\frac{(n-1)!(n-k)}{(n-k)!k!}+\frac{k(n-1)!}{(n-k)!k!}\\
	&=\frac{(n-1)!}{(n-1-k)!k!}+\frac{(n-1)!}{(n-k)!(k-1)!}\\&={{n-1}\choose{k}}+{{n-1}\choose{k-1}} \label{n-1choosek+n-1choosek-1}\\
	&\equiv {{m}\choose{k}}+{{m}\choose{k-1}}, \textrm{ with }m=n-1.
\end{align} 

Observe for a moment that (\ref{n-1choosek+n-1choosek-1}) can be expanding even further. By the same process of going from (\ref{binomial-coefficient-defn}) to (\ref{n-1choosek+n-1choosek-1}), we can say
\begin{align}
	{{n-1}\choose{k-1}} = {{n-2}\choose{k-1}}+{{n-2}\choose{k-2}}.
\end{align}
Thus, (\ref{n-1choosek+n-1choosek-1}) becomes
\begin{align}
	{{n}\choose{k}}={{n-1}\choose{k}}+{{n-2}\choose{k-1}}+{{n-2}\choose{k-2}}.
\end{align}
If we continue this pattern, we can see that we can write the binomial coefficients as a sum of binomial coefficients with incrementally decreasing numerators (i.e n!, (n-1)!, (n-2)!,...). This gives
\begin{align}
	{{n}\choose{k}}&={{n-1}\choose{k}}+{{n-2}\choose{k-1}}+{{n-3}\choose{k-2}}+\cdots+{{1}\choose{k+2-n}}+{{0}\choose{k+1-n}} \\
	&=\sum_{i=1}^{n}{{n-i}\choose{k+1-i}} \\
	&=\sum_{i=0}^{n-1}{{n-1-i}\choose{k-i}}.
\end{align}
Note that we stop the above sequence when the numerator of our factorial sequence has reached zero. If we were to continue the sequence, we would end up having negative factorials in our numerator which would make evaluating the binomial coefficient at that term and the following terms difficult. However, if we happen to have a smaller $k$ than $n$, it may be such that we end up having negative $k$ numbers. This is ok for now, as it will lead to complex infinities in the denominator of our binomial coefficient expressions hence making those terms zero. This will be demonstrated later. Now, suppose we claim the following theorem. 

\begin{theo}[Summing Over All binomial coefficients\label{thm:Summing Over All binomial coefficients}]{1}
	For all $n\in \mathbb{N}$,
	\begin{align}
		\sum_{i=0}^{n}{{n}\choose{i}} =\sum_{i=0}^{n}\frac{n!}{(n-i)!i!}= 2^n.
	\end{align}
\end{theo}

\begin{proof}
	We can then do a proof by induction to prove this is in fact true for all $n,k \in \mathbb{N}$. First, we can see checking the base cases hold as when $n=0$, we have $1=1$, when $n=1$, we have $2=2$, and when $n=3$ we have $4=4$. Next, let's assume for any $n=k$ that (1) holds true. Now, if we let $n=k+1$ we have from the left hand side of our expression,
	\begin{align}
		\sum_{i=0}^{k+1}{{k+1}\choose{i}}&={{k+1}\choose{0}}+{{k+1}\choose{1}}+\cdots+{{k+1}\choose{k}}+{{k+1}\choose{k+1}} \\
		&={{k}\choose{0}}+{{k}\choose{-1}}+{{k}\choose{1}}+{{k}\choose{0}}+\cdots+{{k}\choose{k}}+{{k}\choose{k-1}}+{{k}\choose{k+1}}+{{k}\choose{k}}\\
		&=2{{k}\choose{0}}+2{{k}\choose{1}}+\cdots+2{{k}\choose{k-1}}+2{{k}\choose{k}}\\
		&=2\sum_{i=0}^{k}{{k}\choose{i}}\\
		&=2(2^k)\\
		&=2^{k+1}.
	\end{align}
	Therefore, we can see that for $n=k+1$ that equation (8) holds true, and thus we conclude by induction that (8) holds for all $n\in\mathbb{N}$.
\end{proof} 

Note that in the above proof, we made use of ${{k}\choose{k+1}}={{k}\choose{-1}}=0$. If we were to evaluate each of these using the definition of the binomial coefficient above we may notice a slight issue. Suppose we try to evaluate ${{n}\choose{-1}}$. Using the definition from (1), we would have
\begin{align}
	{{n}\choose{-1}}&=\frac{n!}{(n-(-1))!(-1)!}\\&=\frac{n!}{(n+1)!(-1)!} \\
	&=\frac{n!}{(n)!(n+1)(-1)!} \\
	&=\frac{1}{(n+1)(-1)!}.
\end{align}
From above, we have a negative factorial in the denominator of our expression. Since this is not easily determined as a positive integer factorial would be, we will need to expand this using the Gamma function. 

\begin{defn}[The Gamma function \label{The Gamma function Definition}]{1}
	The gamma function is a mathematical function that generalizes the concept of a factorial to a non-integer, complex number $n$. It is denoted by the Greek letter "$\Gamma$" (gamma) and defined as
	\begin{align}
		\Gamma(n)=(n-1)!\equiv\int_{0}^{\infty}t^{n-1}e^{-t}\dt.
	\end{align}
\end{defn}

Using this gamma function with $n=0$ gives
\begin{align}
	\Gamma(0)&=\int_{0}^{\infty}t^{-1}e^{-t}\dt.
\end{align}
Since $\lim\limits_{t\rightarrow 0^+}t^{-1}e^{-t}=\infty$, we can say the integral under the curve from $0$ to $\infty$ will be divergent, and thus $\infty$. Therefore $\Gamma(0)\equiv\infty$. This allows us to write (18) as
\begin{align}
	\frac{1}{(n+1)(-1)!} = \frac{1}{(n+1)\Gamma(0)}=\lim\limits_{x\rightarrow \infty}\frac{1}{x}=0.
\end{align}
Thus, we can say that ${{n}\choose{-1}}=0$. Similarly, by the same process, if we have ${{n}\choose{n+1}}$ we get
\begin{align}
	{{n}\choose{n+1}}&=\frac{n!}{(n-(n+1))!(n+1)!}\\
	&=\frac{n!}{(-1)!(n)!(n+1)}\\
	&=\frac{1}{(-1)!(n+1)}\\
	&=0
\end{align}

