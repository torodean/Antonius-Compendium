\chapter{General Physics}
\thispagestyle{fancy}

Physics is the scientific study of the fundamental principles that govern the behavior of matter, energy, space, and time. It seeks to understand the natural world by formulating and testing mathematical models and theories based on observations and experimental evidence. Physics is a broad and diverse field that encompasses various sub-disciplines, including classical mechanics, electromagnetism, thermodynamics, quantum mechanics, and relativity, among others. Key areas of study in physics include:

\begin{itemize}
	\item \keyword{Classical Mechanics}: This branch deals with the motion of macroscopic objects and is based on Newton's laws of motion and the concept of conservation of energy and momentum.

	\item \keyword{Electromagnetism}: This field explores the behavior of electric and magnetic fields and their interactions with charged particles, leading to the understanding of electricity, magnetism, and electromagnetic waves.

	\item \keyword{Thermodynamics}: It examines the relationships between heat, work, and energy, providing insights into the behavior of gases, liquids, and solids and the principles of heat engines and refrigeration.

	\item \keyword{Quantum Mechanics}: This revolutionary theory describes the behavior of particles at the atomic and subatomic levels, often challenging our classical intuitions and introducing concepts like wave-particle duality and quantum entanglement.

	\item \keyword{Relativity}: Special and General Relativity, proposed by Albert Einstein, deal with the nature of space, time, and gravity, especially at high speeds and in the presence of massive objects.

	\item \keyword{Nuclear Physics}: It focuses on the study of atomic nuclei, nuclear reactions, and nuclear forces, leading to applications in energy production and understanding the universe's early moments.

	\item \keyword{Particle Physics}: This field investigates the fundamental particles of the universe and their interactions, exploring the subatomic realm with particle accelerators and detectors.

	\item \keyword{Astrophysics}: The study of celestial objects and phenomena, such as stars, galaxies, black holes, and cosmic rays, to understand the structure and evolution of the universe.

Condensed Matter Physics: It examines the properties of solid and liquid matter and studies phenomena like superconductivity, magnetism, and phase transitions.
\end{itemize}

Physics has been crucial in advancing technology and shaping our understanding of the universe, from everyday applications like electronics and telecommunications to groundbreaking discoveries about the fundamental nature of reality. It plays a central role in modern science and continues to be a driving force for innovation and progress in various fields. Physics stands as the foundational and all-encompassing science, profoundly shaping the advancement of all scientific disciplines.

\begin{interestnote}
	A poet once said, “The whole universe is in a glass of wine.” We will probably never know in what sense he meant that, for poets do not write to be understood. But it is true that if we look at a glass of wine closely enough we see the entire universe. There are the things of physics: the twisting liquid which evaporates depending on the wind and weather, the reflections in the glass, and our imagination adds the atoms. The glass is a distillation of the earth’s rocks, and in its composition we see the secrets of the universe’s age, and the evolution of stars. What strange array of chemicals are in the wine? How did they come to be? There are the ferments, the enzymes, the substrates, and the products. There in wine is found the great generalization: all life is fermentation. Nobody can discover the chemistry of wine without discovering, as did Louis Pasteur, the cause of much disease. How vivid is the claret, pressing its existence into the consciousness that watches it! If our small minds, for some convenience, divide this glass of wine, this universe, into parts—physics, biology, geology, astronomy, psychology, and so on—remember that nature does not know it! So let us put it all back together, not forgetting ultimately what it is for. Let it give us one more final pleasure: drink it and forget it all! \cite{bib:feynman lectures}
\end{interestnote}


\section{Measurement}

\keyword{Measurement} is one of the most important tools for studying physics as it relates to a physical reality. A measurement always has some sort of error associated with it and is always relative to \textit{something}.

\begin{quotation}
	``Wherefore relative quantities are not the quantities themselves, whose names they bear, but those sensible measures of them (either accurate or inaccurate), which are commonly used instead of the measured quantities themselves. And if the meaning of words is to be determined [definiendae] by their use, then by the names time, space, place, and motion, their measures [mensurae sensibilies] are properly to be understood; and the expression will be unusual, and purely mathematical, if the measured quantities themselves are meant. On this account, those violate the accuracy of language, which ought to be kept precise, who interpret these words for the measured quantities. Nor do those less defile the purity of mathematical and philosophical truths, who confound real quantities with their relations and sensible measures [vulgaribus mensuris].'' \cite{bib:Newtons Scholium}
\end{quotation}


\section{Chemistry}

\keyword{Inorganic chemistry} is the branch of chemistry that deals with the study of inorganic compounds, which are substances not primarily based on carbon-hydrogen (C-H) bonds. It explores the properties, structures, and behaviors of minerals, metals, gases, and other inorganic substances, playing a crucial role in understanding the fundamental elements and their interactions. The theory of chemistry and chemical reactions, was summarized to a large extent in the periodic chart of \keyword{Mendeleev}, also known as the \keyword{periodic table of elements}. All of these rules were eventually explained by quantum mechanics and thus theoretical chemistry is actually just physics. \keyword{Organic chemistry} is the chemistry of substances which are associated with living things. These are typically complex. This field is closely related to biology.

\keyword{Statistical mechanics} was a branch of physics developed alongside chemistry and has played an extremely important role in history. In any chemical situation, there are a large number of atoms all interacting. If all of these atoms could be analyzed at each collision and kept track of, the phenomenon of chemical reactions could be predicted. Statistical mechanics is the science of heat or thermodynamics which accomplishes this to some degree.

\section{Biology}

\keyword{Biology} is the scientific study of living organisms and their interactions with the environment. It encompasses a wide range of topics, from the molecular and cellular levels to the study of ecosystems and biodiversity. Biology seeks to understand the processes of life, including growth, reproduction, adaptation, and evolution, providing insights into the diversity and complexity of living systems. Biology is composed of many physical phenomenon and systems which can be studies and understood by physics. There are often very intriguing connections between physics and biology. The \keyword{conservation of energy} was in large part inspired by biology when Julius von Mayer demonstrated a connection with the amount of heat taken in and given out by a living creature. Another example is that of nerves which function as very find tubes with positive ions on the outside and negative ions on the inside - much like a capacitor that can be discharged to send signals.

\keyword{Proteins} are large, complex molecules composed of amino acids, essential for various biological functions. They play crucial roles in the structure, function, and regulation of cells and tissues, serving as enzymes, receptors, transporters, and antibodies. \keyword{Enzymes}, a specialized type of protein, act as biological catalysts, speeding up chemical reactions within cells. They are highly specific and crucial for metabolic processes, allowing cells to carry out essential tasks such as digestion, energy production, and DNA replication. A significant achievement since the 1960s has been the precise determination of the atomic arrangement in certain proteins, comprising approximately fifty-six or sixty amino acids in sequence. These studies have located over a thousand atoms, including hydrogen atoms, forming intricate patterns in some proteins, such as \keyword{hemoglobin}. Despite this progress, the mystery remains as to why these patterns function as they do.

\keyword{DNA}, short for deoxyribonucleic acid, is a molecule that carries genetic information in all living organisms. It consists of a double helix structure made up of nucleotide units, each containing a sugar, phosphate group, and a nitrogenous base. DNA serves as the blueprint for the development, functioning, and inheritance of living beings, encoding the instructions required for the synthesis of proteins and controlling various cellular processes. The study of DNA's structure led to the remarkable discovery that it consists of a pair of twisted chains, with specific cross-links between them. These cross-links, represented by adenine, thymine, cytosine, and guanine, fit together in pairs. The pairs form strong bonds and only certain pairs will create these bonds. Therefore, when DNA is split, there is only one way to reproduce the other half in cell reproduction, ensuring the accurate replication of DNA during cell division and passing on specific instructions for protein synthesis to offspring.

No subject or field is making more progress on so many fronts at the present moment, than biology, and if we were to name the most powerful assumption of all, which leads one on and on in an attempt to understand life, it is that all things are made of atoms (see \ref{section:The Atomic Hypothesis}), and that everything that living things do can be understood in terms of the jigglings and wigglings of atoms.

\section{Astronomy}

\keyword{Astronomy} is the scientific study of celestial objects, phenomena, and the universe as a whole. It encompasses the observation, analysis, and understanding of stars, planets, galaxies, and other cosmic structures, as well as the exploration of space and the fundamental principles governing the cosmos. Astronomy plays a crucial role in unraveling the mysteries of the universe, from studying the origins of galaxies to investigating the properties of distant exoplanets and the evolution of the cosmos over billions of years. Astronomy is older than physics and bridged the way for the beginning of physics through observations of the simplistic motion of starts and planets. 

The most remarkable discovery in all of astronomy is that the stars are made of atoms of the same kind as those on the earth. With a \keyword{spectroscope}, we can analyze the frequencies of light waves and can therefore see the presence of atoms that are in different stars. Some chemicals were even discovered in stars before being discovered on earth.

One of the most impressive discoveries was the origin of energy in stars. The nuclear `burning' of hydrogen supplies energy to stars. The stellar properties inside stars make them prime conditions for many nuclear reactions to occur and new elements to be created. The isotopic proportions in our own composition reveal the stellar origins of the elements, suggesting that we were "cooked" in stars and expelled through novae and supernovae explosions. Astronomy, closely linked to physics, offers valuable insights into these processes.

\section{Geology}

\keyword{Geology}, also called \keyword{earth sciences}, is the scientific study of the Earth's structure, composition, and processes that have shaped its history. It investigates the formation of rocks, minerals, and geological features, such as mountains, volcanoes, and earthquakes, as well as the evolution of landscapes and the interactions between the Earth's materials and natural forces. Geology plays a crucial role in understanding Earth's past, present, and future, and it contributes to various fields, including resource exploration, environmental management, and hazard assessment. The question basic to geology is, what makes the earth the way it is?

Meteorology is the scientific study of the Earth's atmosphere and weather phenomena. It closely relates to physics because it involves the application of fundamental physical principles, such as thermodynamics, fluid dynamics, and radiation, to understand and predict atmospheric processes. By analyzing the behavior of air masses, temperature, humidity, pressure, and wind patterns, meteorologists use physics-based models and observations to forecast weather conditions and study atmospheric phenomena like hurricanes, tornadoes, and climate change. The interaction between meteorology and physics plays a crucial role in advancing our understanding of the Earth's atmosphere and its impact on weather patterns and climate dynamics.

While we have considerable knowledge about earthquake waves and the Earth's density distribution, physicists face challenges understanding matter's properties at the immense pressures within the Earth's core. Unlike conditions in stars, the complex mathematics involved in these extreme circumstances remains a hurdle to tackle. Additionally, determining the density and behavior of rocks at high pressure requires experimental investigations, as our theoretical understanding is limited.




\section{Energy}

The \keyword{conservation of energy} is believed to be a fundamental law governing all known natural phenomena that are known to date. It is an exact and unbroken principle, stating that a quantity known as \keyword{energy} remains constant amidst the various changes occurring in nature. This abstract idea is based on mathematical principles, representing a numerical quantity that remains unchanged throughout different events. It is not a description of a concrete mechanism but rather a remarkable fact: the calculated energy value remains the same after observing nature's transformations. Energy has a large number of different forms, and there is a formula for each one. These are: gravitational energy, kinetic energy, heat energy, elastic energy, electrical energy, chemical energy, radiant energy, nuclear energy, mass energy. If we total up the formulas for each of these contributions in a system, it will not change except for energy going in and out of the system.

In quantum mechanics it turns out that the conservation of energy is very closely related to another important property of the world, \textit{things do not depend on the absolute time}.

\subsection{Gravitational Potential Energy}

The general name of energy which has to do with location relative to something else is called \keyword{potential energy}. In the case of objects relative to earth (or other bodies effected by gravitational energy), it is called \keyword{gravitational potential energy}.
	
\begin{defn}[Gravitational Potential Energy \label{Gravitational Potential Energy Definition}]
	Gravitational potential energy $E$ is a form of potential energy associated with the position of an object in a gravitational field. It represents the energy stored in an object due to its height $h$ above a reference point. The higher an object is positioned above the reference point, the greater its gravitational potential energy.
	\begin{align}
		E = mgh,
	\end{align}
	where $m$ is the mass of the object and $g$ is the acceleration due to gravity.
\end{defn}

For static structures and systems, one can apply an imaginary motion to the system (even if it is not really moving or even movable) in order to apply the principal of conservation of energy. This approach is called the \keyword{principle of virtual work}.

\subsection{Kinetic Energy}

The principal of \keyword{motion} gives rise to another type of energy referred to as \keyword{kinetic energy}.

\begin{defn}[Kinetic Energy \label{Kinetic Energy}]
	Kinetic energy $E$ is a fundamental concept in physics that refers to the energy possessed by an object due to its motion. It is the energy associated with the object's velocity and is dependent on both its mass $m$ and the square of the magnitude of its velocity $v$. 
	\begin{align}
		E = \frac{1}{2}mv^2
	\end{align}
\end{defn} 

when associated with relativity, there is a modification to this concept of kinetic energy. An object has energy from simply existing which is referred to as a mass energy which is combined with this kinetic energy of motion.
















