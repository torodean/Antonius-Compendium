/section{Coordinate Systems}

\subsection{Cartesian Coordinates}

\keyword{Cartesian coordinates}, also known as \keyword{rectangular coordinates}, are a system of representing points in a two-dimensional or three-dimensional space using a set of perpendicular axes. Cartesian coordinates can be mathematically extended into higher dimensions.

In two-dimensional Cartesian coordinates, a point is located by specifying its distance (magnitude) along the x-axis and the y-axis from the origin (0,0), where the x-axis is horizontal, and the y-axis is vertical. The coordinates of a point are written as (x, y), where x represents the horizontal distance and y represents the vertical distance. In three-dimensional Cartesian coordinates, a point is located by specifying its distance along the x-axis, y-axis, and z-axis from the origin (0, 0, 0). The coordinates of a point are written as (x, y, z), where x, y, and z represent the distances along the three axes. In higher $n$-dimensions, a similar formulation is used where a point is located by specifying its distance along $n$-axes (typically denoted by the unit vector $\hat{e}_i$) from the origin $\vec{0}$. The coordinates are $(x_0,x_1,...,x_n)$, where $x_i$ reoresebts the distance along the respective $n$ axes.

