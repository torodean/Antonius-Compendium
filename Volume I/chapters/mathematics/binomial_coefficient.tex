\subsection{Pascal's Triangle\label{section:pascals_triangle}}

\keyword{Pascal's Triangle} is a mathematical arrangement of numbers that holds several interesting properties and has applications in various fields. Pascal's Triangle is constructed by starting with a single "1" at the top. Each subsequent row is created by placing "1"s on both ends, and each inner value is the sum of the two values directly above it.
\begin{equation}
\renewcommand{\arraystretch}{1.5}
\begin{tabular}{cccccccccccc}
&      &      &      &      &      &  1   &      &      &      &      &      \\
&      &      &      &      &  1   &      &  1   &      &      &      &      \\
&      &      &      &  1   &      &  2   &      &  1   &      &      &      \\
&      &      &  1   &      &  3   &      &  3   &      &  1   &      &      \\
&      &  1   &      &  4   &      &  6   &      &  4   &      &  1   &      \\
&  1   &      &  5   &      &  10  &      &  10  &      &   5  &      &   1   \\
\end{tabular}\label{pascals_triangle}
\end{equation}
The numbers in Pascal's Triangle are also known as \keyword{binomial coefficients}. The entry in the $n$th row and $k$th column represents the coefficient of the term $x^k$ in the expansion of $(x + 1)^n$. The values in Pascal's Triangle also correspond to combinations (binomial coefficients) that count the number of ways to choose $k$ items from a set of $n$ items without considering the order. Pascal's Triangle is symmetric along its central vertical axis. Each row is a palindrome. The sum of the numbers in each row is a power of 2 (see theorem \ref{thm:Summing Over All binomial coefficients}). In number theory, Pascal's Triangle is involved in finding patterns and relationships in numbers, and it is often connected to the study of the binomial series and power series in calculus. Pascal's Triangle can be extended to include negative row indices and even more complex expansions.

\section{Binomial Coefficients}

\begin{defn}[Binomial Coefficient \label{Binomial Coefficient Definition}]{1}
	The binomial coefficient, ${{n}\choose{k}}$, often denoted as "n choose k," is a mathematical concept used in combinatorics to represent the number of ways to choose k items from a set of n distinct items, regardless of the arrangement of the chosen elements. 	
	\begin{align}
		{{n}\choose{k}}&=\frac{n!}{(n-k)!k!}
	\end{align}
\end{defn}

To begin with, a useful idea is to sum each term of the binomial coefficient which will be used later. First, by definition of the binomial coefficient, we can write
\begin{align}
	{{n}\choose{k}}&=\frac{n!}{(n-k)!k!} \label{binomial-coefficient-defn}\\
	&=\frac{(n-1)!n}{(n-k)!k!}\\&
	=\frac{(n-1)!n-k(n-1)!+k(n-1)!}{(n-k)!k!}\\
	&=\frac{(n-1)!(n-k)}{(n-k)!k!}+\frac{k(n-1)!}{(n-k)!k!}\\
	&=\frac{(n-1)!}{(n-1-k)!k!}+\frac{(n-1)!}{(n-k)!(k-1)!}\\&={{n-1}\choose{k}}+{{n-1}\choose{k-1}} \label{n-1choosek+n-1choosek-1}\\
	&\equiv {{m}\choose{k}}+{{m}\choose{k-1}}, \textrm{ with }m=n-1.
\end{align} 

Observe for a moment that (\ref{n-1choosek+n-1choosek-1}) can be expanding even further. By the same process of going from (\ref{binomial-coefficient-defn}) to (\ref{n-1choosek+n-1choosek-1}), we can say
\begin{align}
	{{n-1}\choose{k-1}} = {{n-2}\choose{k-1}}+{{n-2}\choose{k-2}}.
\end{align}
Thus, (\ref{n-1choosek+n-1choosek-1}) becomes
\begin{align}
	{{n}\choose{k}}={{n-1}\choose{k}}+{{n-2}\choose{k-1}}+{{n-2}\choose{k-2}}.
\end{align}
If we continue this pattern, we can see that we can write the binomial coefficients as a sum of binomial coefficients with incrementally decreasing numerators (i.e n!, (n-1)!, (n-2)!,...). This gives
\begin{align}
	{{n}\choose{k}}&={{n-1}\choose{k}}+{{n-2}\choose{k-1}}+{{n-3}\choose{k-2}}+\cdots+{{1}\choose{k+2-n}}+{{0}\choose{k+1-n}} \\
	&=\sum_{i=1}^{n}{{n-i}\choose{k+1-i}} \\
	&=\sum_{i=0}^{n-1}{{n-1-i}\choose{k-i}}.
\end{align}
Note that we stop the above sequence when the numerator of our factorial sequence has reached zero. If we were to continue the sequence, we would end up having negative factorials in our numerator which would make evaluating the binomial coefficient at that term and the following terms difficult. However, if we happen to have a smaller $k$ than $n$, it may be such that we end up having negative $k$ numbers. This is ok for now, as it will lead to complex infinities in the denominator of our binomial coefficient expressions hence making those terms zero. This will be demonstrated later. Now, suppose we claim the following theorem. 

\begin{theo}[Summing Over All binomial coefficients\label{thm:Summing Over All binomial coefficients}]{1}
	For all $n\in \mathbb{N}$,
	\begin{align}
		\sum_{i=0}^{n}{{n}\choose{i}} =\sum_{i=0}^{n}\frac{n!}{(n-i)!i!}= 2^n.
	\end{align}
\end{theo}

\begin{proof}
	We can then do a proof by induction to prove this is in fact true for all $n,k \in \mathbb{N}$. First, we can see checking the base cases hold as when $n=0$, we have $1=1$, when $n=1$, we have $2=2$, and when $n=3$ we have $4=4$. Next, let's assume for any $n=k$ that (1) holds true. Now, if we let $n=k+1$ we have from the left hand side of our expression,
	\begin{align}
		\sum_{i=0}^{k+1}{{k+1}\choose{i}}&={{k+1}\choose{0}}+{{k+1}\choose{1}}+\cdots+{{k+1}\choose{k}}+{{k+1}\choose{k+1}} \\
		&={{k}\choose{0}}+{{k}\choose{-1}}+{{k}\choose{1}}+{{k}\choose{0}}+\cdots+{{k}\choose{k}}+{{k}\choose{k-1}}+{{k}\choose{k+1}}+{{k}\choose{k}}\\
		&=2{{k}\choose{0}}+2{{k}\choose{1}}+\cdots+2{{k}\choose{k-1}}+2{{k}\choose{k}}\\
		&=2\sum_{i=0}^{k}{{k}\choose{i}}\\
		&=2(2^k)\\
		&=2^{k+1}.
	\end{align}
	Therefore, we can see that for $n=k+1$ that equation (8) holds true, and thus we conclude by induction that (8) holds for all $n\in\mathbb{N}$.
\end{proof} 

Note that in the above proof, we made use of ${{k}\choose{k+1}}={{k}\choose{-1}}=0$. If we were to evaluate each of these using the definition of the binomial coefficient above we may notice a slight issue. Suppose we try to evaluate ${{n}\choose{-1}}$. Using the definition from (1), we would have
\begin{align}
	{{n}\choose{-1}}&=\frac{n!}{(n-(-1))!(-1)!}\\&=\frac{n!}{(n+1)!(-1)!} \\
	&=\frac{n!}{(n)!(n+1)(-1)!} \\
	&=\frac{1}{(n+1)(-1)!}.
\end{align}
From above, we have a negative factorial in the denominator of our expression. Since this is not easily determined as a positive integer factorial would be, we will need to expand this using the Gamma function. 

\begin{defn}[The Gamma function \label{The Gamma function Definition}]{1}
	The gamma function is a mathematical function that generalizes the concept of a factorial to a non-integer, complex number $n$. It is denoted by the Greek letter "$\Gamma$" (gamma) and defined as
	\begin{align}
		\Gamma(n)=(n-1)!\equiv\int_{0}^{\infty}t^{n-1}e^{-t}\dt.
	\end{align}
\end{defn}

Using this gamma function with $n=0$ gives
\begin{align}
	\Gamma(0)&=\int_{0}^{\infty}t^{-1}e^{-t}\dt.
\end{align}
Since $\lim\limits_{t\rightarrow 0^+}t^{-1}e^{-t}=\infty$, we can say the integral under the curve from $0$ to $\infty$ will be divergent, and thus $\infty$. Therefore $\Gamma(0)\equiv\infty$. This allows us to write (18) as
\begin{align}
	\frac{1}{(n+1)(-1)!} = \frac{1}{(n+1)\Gamma(0)}=\lim\limits_{x\rightarrow \infty}\frac{1}{x}=0.
\end{align}
Thus, we can say that ${{n}\choose{-1}}=0$. Similarly, by the same process, if we have ${{n}\choose{n+1}}$ we get
\begin{align}
	{{n}\choose{n+1}}&=\frac{n!}{(n-(n+1))!(n+1)!}\\
	&=\frac{n!}{(-1)!(n)!(n+1)}\\
	&=\frac{1}{(-1)!(n+1)}\\
	&=0
\end{align}
