\chapter{Relativity}
\thispagestyle{fancy}

Relativity refers to two groundbreaking theories developed by Albert Einstein: \keyword{Special Relativity} and \keyword{General Relativity}. Both theories revolutionized our understanding of the universe and how it functions, especially at high speeds and in the presence of strong gravitational fields.

\begin{itemize}
	\item \keyword{Special Relativity}: Introduced in 1905, Special Relativity deals with the behavior of objects moving at constant velocities, especially near the speed of light. It is based on two postulates: the principle of relativity (laws of physics are the same for all inertial observers) and the constancy of the speed of light in a vacuum. Key principles of Special Relativity include:
	\begin{itemize}
		\item \keyword{Time Dilation}: Moving clocks appear to run slower compared to stationary clocks from the perspective of an observer at rest.
		\item \keyword{Length Contraction}: Moving objects appear shorter in the direction of motion relative to a stationary observer.
		\item \keyword{Mass-Energy Equivalence}: $E=mc^2$, where "E" is energy, "m" is mass, and "c" is the speed of light. This famous equation shows that mass and energy are interchangeable.
	\end{itemize}

	\item \keyword{General Relativity}: Formulated in 1915, General Relativity is a theory of gravity that describes the curvature of spacetime caused by the presence of mass and energy. Unlike Newtonian gravity, which considers gravity as an attractive force between masses, General Relativity attributes gravity to the curvature of spacetime caused by massive objects. Key principles of General Relativity include:
	\begin{itemize}
		\item \keyword{Curved Spacetime}: Massive objects like stars and planets curve the fabric of spacetime around them, and other objects move along the curved paths in response to this curvature.
		\item \keyword{Gravitational Time Dilation}: Clocks in stronger gravitational fields (e.g., near massive objects) run slower compared to clocks in weaker fields.
		\item \keyword{Gravitational Waves}: General Relativity predicts the existence of gravitational waves, ripples in spacetime caused by violent cosmic events.
	\end{itemize}
\end{itemize}

Both Special and General Relativity have been confirmed through numerous experiments and observations, and they have far-reaching implications for our understanding of the cosmos. Special Relativity's effects become significant at high speeds, near the speed of light, while General Relativity is essential for describing gravity and the behavior of massive objects, such as stars, galaxies, and black holes. These theories have not only revolutionized fundamental physics but have also influenced technology, astronomy, and our perception of the nature of space, time, and the universe.