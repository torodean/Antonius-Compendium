\section{Integrals}

The term "\keyword{integral}" encompasses several distinct concepts in mathematics. Its most prevalent meaning relates to the fundamental concept in calculus, where it involves the process of summing infinitesimally small components to determine the content of a continuous region. In calculus, an integral is a fundamental mathematical concept that can be interpreted as measuring area or extending the notion of area to more complex shapes. Alongside derivatives, integrals are the cornerstone of calculus. They are also referred to as \keyword{antiderivatives} or \keyword{primitives}. The act of evaluating an integral is known as integration (a term sometimes replaced by the more historical word "quadrature"). When an integral is approximated numerically, it is called numerical integration.

\begin{defn}[The Antiderivative]{1}
A function $F$ is called an \keyword{antiderivative} (or \keyword{integral}) of a function $f$ on a given open interval if $\frac{d}{dx}F(x)=f(x)$ for all $x$ in the interval. This is typically denoted
\begin{align}
f(x)=\frac{d}{dx}F(x) \implies \int_{a}^{b} f(x)dx = \int_a^b \frac{d}{dx}F(x) dx = F(x)\Big|_{a}^{b} = F(b)-F(a)
\end{align}
\end{defn}

%\begin{theo}[The Fundamental Theorem Of Calculus]{1}
%Suppose that $f(x)$ is continuous on the interval [a,b]. If $F(x)$ is any antiderivative of $f(x)$, then $\int_a^b f(x) dx = F(b)-F(a)$. Given $F(x) = \int_a^b f(x) dx$, then $\frac{d}{dx}F(x) = f(x)$.
%\end{theo}
