\section{Energy}

The \keyword{conservation of energy} is believed to be a fundamental law governing all known natural phenomena that are known to date. It is an exact and unbroken principle, stating that a quantity known as \keyword{energy} remains constant amidst the various changes occurring in nature. This abstract idea is based on mathematical principles, representing a numerical quantity that remains unchanged throughout different events. It is not a description of a concrete mechanism but rather a remarkable fact: the calculated energy value remains the same after observing nature's transformations. Energy has a large number of different forms, and there is a formula for each one. These are: gravitational energy, kinetic energy, heat energy, elastic energy, electrical energy, chemical energy, radiant energy, nuclear energy, mass energy. If we total up the formulas for each of these contributions in a system, it will not change except for energy going in and out of the system.

\subsection{Gravitational Potential Energy}

The general name of energy which has to do with location relative to something else is called \keyword{potential energy}. In the case of objects relative to earth (or other bodies effected by gravitational energy), it is called \keyword{gravitational potential energy}.
	
\begin{defn}[Gravitational Potential Energy \label{Gravitational Potential Energy Definition}]
	Gravitational potential energy $E$ is a form of potential energy associated with the position of an object in a gravitational field. It represents the energy stored in an object due to its height $h$ above a reference point. The higher an object is positioned above the reference point, the greater its gravitational potential energy.
	\begin{align}
		E = mgh,
	\end{align}
	where $m$ is the mass of the object and $g$ is the acceleration due to gravity.
\end{defn}

For static structures and systems, one can apply an imaginary motion to the system (even if it is not really moving or even movable) in order to apply the principal of conservation of energy. This approach is called the \keyword{principle of virtual work}.

\subsection{Kinetic Energy}

The principal of \keyword{motion} gives rise to another type of energy referred to as \keyword{kinetic energy}.

\begin{defn}[Kinetic Energy \label{Kinetic Energy}]
	Kinetic energy $E$ is a fundamental concept in physics that refers to the energy possessed by an object due to its motion. It is the energy associated with the object's velocity and is dependent on both its mass $m$ and the square of the magnitude of its velocity $v$. 
	\begin{align}
		E = \frac{1}{2}mv^2
	\end{align}
\end{defn} 