\section{Biology}

\keyword{Biology} is the scientific study of living organisms and their interactions with the environment. It encompasses a wide range of topics, from the molecular and cellular levels to the study of ecosystems and biodiversity. Biology seeks to understand the processes of life, including growth, reproduction, adaptation, and evolution, providing insights into the diversity and complexity of living systems. Biology is composed of many physical phenomenon and systems which can be studies and understood by physics. There are often very intriguing connections between physics and biology. The \keyword{conservation of energy} was in large part inspired by biology when Julius von Mayer demonstrated a connection with the amount of heat taken in and given out by a living creature. Another example is that of nerves which function as very find tubes with positive ions on the outside and negative ions on the inside - much like a capacitor that can be discharged to send signals.

\keyword{Proteins} are large, complex molecules composed of amino acids, essential for various biological functions. They play crucial roles in the structure, function, and regulation of cells and tissues, serving as enzymes, receptors, transporters, and antibodies. \keyword{Enzymes}, a specialized type of protein, act as biological catalysts, speeding up chemical reactions within cells. They are highly specific and crucial for metabolic processes, allowing cells to carry out essential tasks such as digestion, energy production, and DNA replication. A significant achievement since the 1960s has been the precise determination of the atomic arrangement in certain proteins, comprising approximately fifty-six or sixty amino acids in sequence. These studies have located over a thousand atoms, including hydrogen atoms, forming intricate patterns in some proteins, such as \keyword{hemoglobin}. Despite this progress, the mystery remains as to why these patterns function as they do.

\keyword{DNA}, short for deoxyribonucleic acid, is a molecule that carries genetic information in all living organisms. It consists of a double helix structure made up of nucleotide units, each containing a sugar, phosphate group, and a nitrogenous base. DNA serves as the blueprint for the development, functioning, and inheritance of living beings, encoding the instructions required for the synthesis of proteins and controlling various cellular processes. The study of DNA's structure led to the remarkable discovery that it consists of a pair of twisted chains, with specific cross-links between them. These cross-links, represented by adenine, thymine, cytosine, and guanine, fit together in pairs. The pairs form strong bonds and only certain pairs will create these bonds. Therefore, when DNA is split, there is only one way to reproduce the other half in cell reproduction, ensuring the accurate replication of DNA during cell division and passing on specific instructions for protein synthesis to offspring.

No subject or field is making more progress on so many fronts at the present moment, than biology, and if we were to name the most powerful assumption of all, which leads one on and on in an attempt to understand life, it is that all things are made of atoms (see \ref{section:The Atomic Hypothesis}), and that everything that living things do can be understood in terms of the jigglings and wigglings of atoms.