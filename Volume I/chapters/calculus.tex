\chapter{Calculus}
\thispagestyle{fancy}

\keyword{Calculus} is a branch of mathematics that deals with the study of change and motion. It encompasses two main components: differentiation and integration.

\begin{itemize}
	\item \keyword{Differentiation}: This involves finding the rate at which a quantity changes with respect to another variable. The derivative is a fundamental concept in calculus, representing the instantaneous rate of change of a function at a particular point. It helps analyze the behavior of functions, such as finding slopes of curves, determining maximum and minimum points, and understanding the concept of velocity and acceleration.
	
	\item \keyword{Integration}: Integration is the reverse process of differentiation. It involves calculating the accumulation or total amount of a changing quantity over an interval. The integral of a function represents the area under the curve of the function over a given range. It is used to solve problems involving areas, volumes, and quantities related to accumulation, such as calculating total distance traveled from a velocity function or finding the area of a region bounded by a curve.
\end{itemize}

Calculus has numerous real-world applications, ranging from physics and engineering to economics and biology. It provides essential tools for understanding how things change, predicting behavior, and solving complex problems that involve continuous change and motion. Developed independently by Isaac Newton and Gottfried Wilhelm Leibniz in the 17th century, calculus remains a fundamental and powerful branch of mathematics used in various scientific and practical domains.