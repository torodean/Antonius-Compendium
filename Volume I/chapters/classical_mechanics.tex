\chapter{Classical Mechanics}
\thispagestyle{fancy}

\keyword{Classical mechanics} is a branch of physics that describes the motion of objects and systems under the influence of forces. It forms the foundation of mechanics before the advent of quantum mechanics and relativistic physics. Classical mechanics is based on Newton's laws of motion and the concept of conservation of energy and momentum. Key principles and concepts of classical mechanics include:

\begin{itemize}
	\item \keyword{Newton's Laws of Motion}: Sir Isaac Newton formulated three fundamental laws that govern the motion of objects. The first law (Law of Inertia) states that an object at rest remains at rest, and an object in motion continues to move at a constant velocity unless acted upon by an external force. The second law describes how the acceleration of an object is directly proportional to the net force applied and inversely proportional to its mass. The third law states that for every action, there is an equal and opposite reaction.
	
	\item \keyword{Conservation of Energy}: The principle of energy conservation states that the total energy of an isolated system remains constant over time. Energy can transform from one form to another (e.g., kinetic energy to potential energy), but the total amount of energy remains unchanged.
	
	\item \keyword{Conservation of Momentum}: The principle of momentum conservation states that the total momentum of an isolated system remains constant, provided no external forces act on it. Momentum is the product of an object's mass and velocity.
	
	\item \keyword{Gravitation}: Classical mechanics includes the study of gravitational forces between objects, described by Newton's law of universal gravitation. This law explains how objects attract each other with a force proportional to their masses and inversely proportional to the square of the distance between them.
	
	\item \keyword{Harmonic Motion}: The study of harmonic motion involves oscillations and vibrations of systems, such as a pendulum or a mass-spring system. These motions follow simple harmonic motion equations and exhibit periodic behavior.
\end{itemize}

Classical mechanics provides accurate and practical predictions for a wide range of everyday scenarios and macroscopic systems. While it is highly effective in describing the behavior of objects at non-relativistic speeds, it becomes less accurate when dealing with extremely high speeds or microscopic particles, where quantum mechanics and relativistic physics are more appropriate. Nonetheless, classical mechanics remains a crucial and fundamental branch of physics, forming the basis for understanding the motion of everyday objects and engineering applications.


\section{Gravity}

The law of gravitation is that every object in the universe attracts every other object with a force which for any two bodies is proportional to the mass of each and varies inversely as the square of the distance between them. This is typically mathematically represented as

\begin{align}
F = \frac{Gm_1m_2}{r^2}
\end{aling}

While we can describe how gravity works mathematically, its underlying mechanism remains somewhat unknown. The absence of a known mechanism drives scientific inquiry, leading to further discoveries. No mechanism has been devised to "explain" gravity without simultaneously predicting the existence of some other phenomenon. To a great precision, this force is exactly proportional to mass, and therefore fundamentally related to \keyword{inertia}.

\begin{questions}
	\item Is this law a constant through time - perhaps the constant changes?
	\item Does this law somehow relate to the force of electrical attraction since they are both proportional to the inverse of the distance squared?
\end{questions}

\begin{interestote}
	The relative strengths of electrical and gravitational interactions between two electrons is
 	\begin{align}
  		\frac{\text{Gravitation Attraction}}{\text{Electrical Repulsion}} = \frac{1}{4.17\times 10^{42}}.
  	\end{align}
   	If we compare the time it takes for light to go across a proton ($10^{-24}$ seconds) to the age of the universe, which is approximately $2\times 10^{10}$ years, the result is $10^{-42}$. They both have a similar number of zeros, leading to a potential idea that the gravitational constant may be linked to the age of the universe.
\end{interestote}

\subsection{Keplar's Laws}

Johannes Kepler was a German mathematician, astronomer, and astrologer who made significant contributions to our understanding of the solar system and planetary motion during the late 16th and early 17th centuries. Kepler is best known for formulating three fundamental laws of planetary motion, which laid the groundwork for modern astronomy. These laws were based on careful observations made by Tycho Brahe. \keyword{Kepler's laws} are
\begin{enumerate}
	\item Each planet moves around the sun in an ellipse, with the sun at one focus.
 	\item The radius vector from the sun to the planet sweeps out equal areas in equal intervals of time.
  	\item The squares of the periods of any two planets are proportional to the cubes of the semimajor axes of their respective orbits: $T\propto a^{3/2}$.
\end{enumerate}

Galileo was studying these laws when he discovered the principal of \keyword{inertia}-\textit{if something is moving, with nothing touching it and completely undisturbed, it will go on forever, coasting at a uniform speed in a straight line.}

\begin{questions}
	\item What is the cause of intertia?
 	\item Does an objects inertia lower over time?
\end{questions}

Newton further extended this by introducing the idea of a \keyword{force}. If a body is to change speed or direction, a force must be applied to it in the direction of the change. This gives the understanding that there must be a force acting on planets in order to keep them in rotation around the sun - rather than just flying off into space.








\section{Motion}

keyword{Motion} is the change in the position of an object relative to a reference point over time. It involves the displacement of an object from one location to another in a continuous manner. Motion can occur in various forms, such as \keyword{linear motion} (movement along a straight line), \keyword{circular motion} (movement in a circular path), \keyword{rotational motion} (movement around a fixed axis), or \keyword{oscillatory motion} (back-and-forth movement around a central position). 

\keyword{Velocity} is a vector quantity that describes the rate of change of an object's position with respect to time. It specifies both the speed and the direction of an object's motion. Velocity $\vec{v}$ is calculated by dividing the displacement (change in position) $\Delta x$ of the object by the time taken to cover that displacement $\Delta t$. Velocity is expressed in units of distance per unit of time, such as meters per second (m/s) or kilometers per hour (km/h). If the velocity is constant over time, it is referred to as constant velocity, and if the velocity is changing, it is referred to as variable velocity. When the direction of an object's motion remains constant, the velocity is considered to be uniform; otherwise, it is non-uniform. Velocity plays a crucial role in describing and analyzing the motion of objects in physics and other scientific disciplines.
\begin{align}
\vec{v}(t)=\lim_{\Delta t \arrow 0}\frac{\Delta \vec{x}(t)}{\Delta t} = \frac{d\vec{x}(t)}{dt} \implies $\vec{x}(t) = \int \vec{v}(t) dt
\end{align}

\keyword{Acceleration} is a vector quantity that describes the rate of change of an object's velocity with respect to time. It indicates how quickly an object's velocity is changing or how much the object is speeding up or slowing down. Acceleration is defined as the change in velocity divided by the change in time. Acceleration, like velocity, is a vector quantity, meaning it has both magnitude and direction. If an object is moving in a straight line, acceleration is in the same direction as the net force acting on the object. If the object is moving in a curved path, the acceleration is directed towards the center of the curvature. The units of acceleration are typically expressed as distance per unit time squared, such as meters per second squared (m/s²) or centimeters per second squared (cm/s²).
\begin{align}
\vec{a}(t)=\lim_{\Delta t \arrow 0}\frac{\Delta \vec{v}(t)}{\Delta t} = \frac{d\vec{v}(t)}{dt} = \frac{d^2\vec{x}(t)}{dt^2} \implies $\vec{v}(t) = \int \vec{a}(t) dt
\end{align}

\keyword{Jerk} is a vector quantity that represents the rate of change of acceleration with respect to time. In other words, jerk measures how quickly an object's acceleration is changing or how rapidly the object's rate of change in velocity is varying. Jerk is a vector quantity, just like acceleration and velocity, meaning it has both magnitude and direction. However, jerk is less commonly used in everyday physics problems compared to acceleration, which is more frequently encountered. In most cases, acceleration is sufficient to describe the motion of objects, and jerk is not a significant factor in standard physics analysis. Nonetheless, jerk becomes relevant in certain specialized fields, such as robotics, aerospace engineering, and studies involving rapid changes in motion or acceleration profiles.
\begin{align}
\vec{j}(t)=\lim_{\Delta t \arrow 0}\frac{\Delta \vec{a}(t)}{\Delta t} = \frac{d\vec{a}(t)}{dt} = \frac{d^2\vec{v}(t)}{dt^2} = \frac{d^3\vec{x}(t)}{dt^3} \implies $\vec{a}(t) = \int \vec{j}(t) dt
\end{align}


\subsection{Newton's Laws}

\keyword{Momentum} $\vec{p}$ is a vector quantity that measures the quantity of motion possessed by an object. It is the product of an object's mass and its velocity and is a fundamental concept in classical mechanics. In this context, \keyword{Mass} is a quantitative measure of inertia. Since momentum is a vector, it has both magnitude and direction. The direction of the momentum vector is the same as the direction of the velocity vector. The SI unit of momentum is kilogram-meter per second (kg·m/s).

\begin{align}
	\vec{p} = m\vec{v}
\end{align}

Following Galileo's discovery of the principal of \keyword{inertia}, Newton set out to describe the rules for findong out how an object changes its \keyword{motion} when effected by something.

\begin{enumerate}
	\item An object at rest will remain at rest, and an object in motion will continue to move with a constant velocity in a straight line unless acted upon by an external force./footnote{This is simply a restating of Galileo's described principal of inertia.}
 	\item The time-rate-of-change of a quantity called momentum is proportional to the force.
  	\begin{align}
   		\vec{F}=\frac{d\vec{p}}{dt} = m\frac{d\vec{v}}{dt} = m\frac{d^2\vec{x}}{dt^2} = m\vec{a}
   	\end{align}
  	\item For every action, there is an equal and opposite reaction.
\end{enumerate}


