\chapter{Classical Mechanics}
\thispagestyle{fancy}

\keyword{Classical mechanics} is a branch of physics that describes the motion of objects and systems under the influence of forces. It forms the foundation of mechanics before the advent of quantum mechanics and relativistic physics. Classical mechanics is based on Newton's laws of motion and the concept of conservation of energy and momentum. Key principles and concepts of classical mechanics include:

\begin{itemize}
	\item \keyword{Newton's Laws of Motion}: Sir Isaac Newton formulated three fundamental laws that govern the motion of objects. The first law (Law of Inertia) states that an object at rest remains at rest, and an object in motion continues to move at a constant velocity unless acted upon by an external force. The second law describes how the acceleration of an object is directly proportional to the net force applied and inversely proportional to its mass. The third law states that for every action, there is an equal and opposite reaction.
	
	\item \keyword{Conservation of Energy}: The principle of energy conservation states that the total energy of an isolated system remains constant over time. Energy can transform from one form to another (e.g., kinetic energy to potential energy), but the total amount of energy remains unchanged.
	
	\item \keyword{Conservation of Momentum}: The principle of momentum conservation states that the total momentum of an isolated system remains constant, provided no external forces act on it. Momentum is the product of an object's mass and velocity.
	
	\item \keyword{Gravitation}: Classical mechanics includes the study of gravitational forces between objects, described by Newton's law of universal gravitation. This law explains how objects attract each other with a force proportional to their masses and inversely proportional to the square of the distance between them.
	
	\item \keyword{Harmonic Motion}: The study of harmonic motion involves oscillations and vibrations of systems, such as a pendulum or a mass-spring system. These motions follow simple harmonic motion equations and exhibit periodic behavior.
\end{itemize}

Classical mechanics provides accurate and practical predictions for a wide range of everyday scenarios and macroscopic systems. While it is highly effective in describing the behavior of objects at non-relativistic speeds, it becomes less accurate when dealing with extremely high speeds or microscopic particles, where quantum mechanics and relativistic physics are more appropriate. Nonetheless, classical mechanics remains a crucial and fundamental branch of physics, forming the basis for understanding the motion of everyday objects and engineering applications.