\chapter{Classical Mechanics}
\thispagestyle{fancy}

\keyword{Classical mechanics} is a branch of physics that describes the motion of objects and systems under the influence of forces. It forms the foundation of mechanics before the advent of quantum mechanics and relativistic physics. Classical mechanics is based on Newton's laws of motion and the concept of conservation of energy and momentum. Key principles and concepts of classical mechanics include:

\begin{itemize}
	\item \keyword{Newton's Laws of Motion}: Sir Isaac Newton formulated three fundamental laws that govern the motion of objects. The first law (Law of Inertia) states that an object at rest remains at rest, and an object in motion continues to move at a constant velocity unless acted upon by an external force. The second law describes how the acceleration of an object is directly proportional to the net force applied and inversely proportional to its mass. The third law states that for every action, there is an equal and opposite reaction.
	
	\item \keyword{Conservation of Energy}: The principle of energy conservation states that the total energy of an isolated system remains constant over time. Energy can transform from one form to another (e.g., kinetic energy to potential energy), but the total amount of energy remains unchanged.
	
	\item \keyword{Conservation of Momentum}: The principle of momentum conservation states that the total momentum of an isolated system remains constant, provided no external forces act on it. Momentum is the product of an object's mass and velocity.
	
	\item \keyword{Gravitation}: Classical mechanics includes the study of gravitational forces between objects, described by Newton's law of universal gravitation. This law explains how objects attract each other with a force proportional to their masses and inversely proportional to the square of the distance between them.
	
	\item \keyword{Harmonic Motion}: The study of harmonic motion involves oscillations and vibrations of systems, such as a pendulum or a mass-spring system. These motions follow simple harmonic motion equations and exhibit periodic behavior.
\end{itemize}

Classical mechanics provides accurate and practical predictions for a wide range of everyday scenarios and macroscopic systems. While it is highly effective in describing the behavior of objects at non-relativistic speeds, it becomes less accurate when dealing with extremely high speeds or microscopic particles, where quantum mechanics and relativistic physics are more appropriate. Nonetheless, classical mechanics remains a crucial and fundamental branch of physics, forming the basis for understanding the motion of everyday objects and engineering applications.


\section{Gravity}

The law of gravitation is that every object in the universe attracts every other object with a force which for any two bodies is proportional to the mass of each and varies inversely as the square of the distance between them. This is typically mathematically represented as

\begin{align}
F = \frac{Gm_1m_2}{r^2}
\end{aling}

While we can describe how gravity works mathematically, its underlying mechanism remains somewhat unknown. The absence of a known mechanism drives scientific inquiry, leading to further discoveries. No mechanism has been devised to "explain" gravity without simultaneously predicting the existence of some other phenomenon. To a great precision, this force is exactly proportional to mass, and therefore fundamentally related to \keyword{inertia}.

\begin{questions}
	\item Is this law a constant through time - perhaps the constant changes?
	\item Does this law somehow relate to the force of electrical attraction since they are both proportional to the inverse of the distance squared?
\end{questions}

\begin{interestote}
	The relative strengths of electrical and gravitational interactions between two electrons is
 	\begin{align}
  		\frac{\text{Gravitation Attraction}}{\text{Electrical Repulsion}} = \frac{1}{4.17\times 10^{42}}.
  	\end{align}
   	If we compare the time it takes for light to go across a proton ($10^{-24}$ seconds) to the age of the universe, which is approximately $2\times 10^{10}$ years, the result is $10^{-42}$. They both have a similar number of zeros, leading to a potential idea that the gravitational constant may be linked to the age of the universe.
\end{interestote}

\subsection{Keplar's Laws}

Johannes Kepler was a German mathematician, astronomer, and astrologer who made significant contributions to our understanding of the solar system and planetary motion during the late 16th and early 17th centuries. Kepler is best known for formulating three fundamental laws of planetary motion, which laid the groundwork for modern astronomy. These laws were based on careful observations made by Tycho Brahe. \keyword{Kepler's laws} are
\begin{enumerate}
	\item Each planet moves around the sun in an ellipse, with the sun at one focus.
 	\item The radius vector from the sun to the planet sweeps out equal areas in equal intervals of time.
  	\item The squares of the periods of any two planets are proportional to the cubes of the semimajor axes of their respective orbits: $T\propto a^{3/2}$.
\end{enumerate}

Galileo was studying these laws when he discovered the principal of \keyword{inertia}-\textit{if something is moving, with nothing touching it and completely undisturbed, it will go on forever, coasting at a uniform speed in a straight line.}

\begin{questions}
	\item What is the cause of intertia?
 	\item Does an objects inertia lower over time?
\end{questions}

Newton further extended this by introducing the idea of a \keyword{force}. If a body is to change speed or direction, a force must be applied to it in the direction of the change. This gives the understanding that there must be a force acting on planets in order to keep them in rotation around the sun - rather than just flying off into space.




