\chapter{Statistical Mechanics \& Thermodynamics}
\thispagestyle{fancy}

\keyword{Statistical mechanics} is a branch of physics that aims to explain the macroscopic properties of a system (such as temperature, pressure, and entropy) by understanding the behavior and interactions of its microscopic constituents at the atomic or molecular level. It bridges the gap between the microscopic world of particles and the macroscopic world we observe in everyday life. Key concepts and principles of statistical mechanics include:

\begin{itemize} 
	\item \keyword{Microstates} and \keyword{Macrostates}: In statistical mechanics, a system's microstate refers to the specific arrangement and energy distribution of its individual particles. A macrostate, on the other hand, represents the observable properties of the system, such as its temperature, pressure, and volume. The goal of statistical mechanics is to determine the probabilities of different microstates leading to a given macrostate.
	
	\item \keyword{Ensembles}: Statistical mechanics uses ensembles, which are collections of similar systems, to study statistical properties. Common ensembles include the microcanonical ensemble (isolated system with constant energy), canonical ensemble (system in thermal contact with a heat reservoir), and grand canonical ensemble (system with exchange of energy and particles with a heat reservoir).
	
	\item \keyword{Boltzmann Distribution}: The Boltzmann distribution relates the probabilities of different energy states to their corresponding energies and the system's temperature. It allows us to predict how particles are distributed among different energy levels in a system.
	
	\item \keyword{Entropy} and Entropy Maximization: Entropy is a fundamental concept in statistical mechanics, representing the measure of a system's disorder or randomness. The second law of thermodynamics states that isolated systems tend to evolve toward states of higher entropy. Statistical mechanics provides a statistical interpretation of entropy and explains the tendency of systems to maximize their entropy over time.
	
	\item \keyword{Statistical Thermodynamics}: Statistical mechanics connects with thermodynamics, relating the macroscopic thermodynamic properties (e.g., internal energy, temperature, and heat capacity) to the statistical properties of microscopic constituents. This connection allows us to derive thermodynamic quantities from the statistical behavior of particles.
\end{itemize}

Statistical mechanics has wide-ranging applications in various scientific disciplines, including physics, chemistry, biology, and materials science. It is essential for understanding phase transitions, chemical reactions, and the behavior of matter under different conditions. Additionally, statistical mechanics forms the basis for exploring complex systems, such as gases, liquids, and solids, and plays a crucial role in the development of many modern technologies.

\section{The Atomic Hypothesis}

The \keyword{atomic hypothesis} is a framework for how matter works on an atomic level.

\begin{quotation}
	``All things are made of atoms—little particles that move around in perpetual motion, attracting each other when they are a little distance apart, but repelling upon being squeezed into one another.''
\end{quotation}

Many questions can arise from this.

\begin{enumerate}
	\item Where do these atoms get their energy from?
 	\item Where do these atoms get their inertia from?
  	\item Do these atoms transfer energy among eachother?
   	\item Are these atoms within some medium with which they can interact?
\end{enumerate}

Atoms are on the scale of Angstroms\footnote{An angstrom, symbolized as ``\AA``, is a unit of length used primarily in the fields of chemistry, physics, and nanotechnology to measure atomic and molecular distances, as well as the size of atoms and molecules. One angstrom is equal to 0.1 nanometers or $10^{-10}$ meters, representing a very small scale typically used to describe atomic and molecular dimensions.}. 
