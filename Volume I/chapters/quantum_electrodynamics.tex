\chapter{Quantum Electrodynamics}
\thispagestyle{fancy}

\keyword{Quantum Electrodynamics} (QED) is a quantum field theory that describes the electromagnetic force and its interactions with charged particles, such as electrons and photons. It is considered one of the most successful and accurate scientific theories ever developed and is a cornerstone of modern particle physics. Key aspects and principles of Quantum Electrodynamics include:

\begin{itemize}
	\item \keyword{Quantization of Electromagnetic Fields}: QED treats the electromagnetic field as a quantum field, where photons are the quantized particles representing the discrete packets of electromagnetic energy.

	\item \keyword{Feynman Diagrams}: Feynman diagrams are a powerful tool used in QED to visualize and calculate particle interactions. They depict the exchange of photons between charged particles, enabling precise predictions of scattering processes and particle interactions.

	\item \keyword{Renormalization}: QED encounters infinities in certain calculations due to the self-interactions of charged particles with their own electromagnetic fields. Renormalization is a technique used to remove these infinities, allowing meaningful and accurate predictions.

	\item \keyword{Quantum Loops}: In QED, particles can interact with each other through quantum loops, where virtual particles (e.g., virtual photons) briefly pop into existence and influence the interactions between charged particles.

	\item \keyword{Gauge Invariance}: QED exhibits gauge invariance, meaning that different mathematical representations of the theory lead to physically equivalent results. This property ensures that observable quantities are independent of the choice of mathematical description.

	\item \keyword{Electron Self-Energy}: QED accounts for the electron's self-energy, which arises from its interaction with its own electromagnetic field. This self-energy correction leads to subtle shifts in electron properties, such as its mass and magnetic moment.
\end{itemize}

Quantum Electrodynamics has been extensively tested through precise experiments and is one of the most accurate and well-validated theories in physics. It successfully explains a wide range of phenomena, including electromagnetic interactions between charged particles, the behavior of light and photons, and the fine structure of atomic spectra. Moreover, QED has been instrumental in the development of other quantum field theories, such as Quantum Chromodynamics (QCD) and the electroweak theory, which describe the strong and weak nuclear forces.