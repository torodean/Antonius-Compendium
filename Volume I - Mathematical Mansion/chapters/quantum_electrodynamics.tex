\chapter{Quantum Electrodynamics}
\thispagestyle{fancy}

\keyword{Quantum Electrodynamics} (QED) is a quantum field theory that describes the electromagnetic force and its interactions with charged particles, such as electrons and photons. It is considered one of the most successful and accurate scientific theories ever developed and is a cornerstone of modern particle physics. Key aspects and principles of Quantum Electrodynamics include:

\begin{itemize}
	\item \keyword{Quantization of Electromagnetic Fields}: QED treats the electromagnetic field as a quantum field, where photons are the quantized particles representing the discrete packets of electromagnetic energy.

	\item \keyword{Feynman Diagrams}: Feynman diagrams are a powerful tool used in QED to visualize and calculate particle interactions. They depict the exchange of photons between charged particles, enabling precise predictions of scattering processes and particle interactions.

	\item \keyword{Renormalization}: QED encounters infinities in certain calculations due to the self-interactions of charged particles with their own electromagnetic fields. Renormalization is a technique used to remove these infinities, allowing meaningful and accurate predictions.

	\item \keyword{Quantum Loops}: In QED, particles can interact with each other through quantum loops, where virtual particles (e.g., virtual photons) briefly pop into existence and influence the interactions between charged particles.

	\item \keyword{Gauge Invariance}: QED exhibits gauge invariance, meaning that different mathematical representations of the theory lead to physically equivalent results. This property ensures that observable quantities are independent of the choice of mathematical description.

	\item \keyword{Electron Self-Energy}: QED accounts for the electron's self-energy, which arises from its interaction with its own electromagnetic field. This self-energy correction leads to subtle shifts in electron properties, such as its mass and magnetic moment.
\end{itemize}

Quantum Electrodynamics has been extensively tested through precise experiments and is one of the most accurate and well-validated theories in physics. It successfully explains a wide range of phenomena, including electromagnetic interactions between charged particles, the behavior of light and photons, and the fine structure of atomic spectra. Moreover, QED has been instrumental in the development of other quantum field theories, such as Quantum Chromodynamics (QCD) and the electroweak theory, which describe the strong and weak nuclear forces.


\section{Introduction}

In this one theory we have the basic rules for all ordinary phenomena except for gravitation and nuclear processes. At the present time no exceptions are found to the quantum-electrodynamic laws outside the nucleus, and there we do not know whether there is an exception because we simply do not know what is going on in the nucleus. This theory predicted a lot of interesting things, such as very high energy photons, gamma rays, etc. Another remarkable prediction is that of antiparticles. These are particles of the same mass, but opposite charge. For example,  the opposite of the electron is the positron and these two coming together would annihilate each-other with the emission of electromagnetic radiation. The generalization of this, that for each particle there is an antiparticle, turns out to be true.

Just as electrical interactions can be connected to a photon, Hideki Yukawa suggested that the forces between neutrons and protons also have a field of some kind, and that when this field jiggles, it behaves like a particle. The characteristics were able to be deduced for this particle and experimentation was done. Around 1947, other particles were found, the \keyword{$\pi$-meson} (or \keyword{pion}). After this, more particles were discovered in cosmic rays and elsewhere which are distinct. We do not today understand these various particles as different aspects of the same thing, and the fact that we have so many unconnected particles is a representation of the fact that we have so much unconnected information without a good theory. In an attempt to classify and categorize these new particles similar to how elements are categorized in the periodic table, a new number (similar in concept to the electric charge) can be assigned to each particle, called its ``\keyword{strangeness},'' $S$. This number is conserved, like the electric charge, in reactions which take place by nuclear forces. 

\begin{table}[htbp]
	\centering
	\label{tab:strangeness-classification}
	\begin{tabular}{lcccc}
		\toprule
		\textbf{Particle} & \textbf{Mass (MeV)} & \textbf{Charge (e)} & \textbf{Strangeness} & \textbf{Category} \\
		\midrule
		Proton & 938.272 & $+1$ & $0$ & Baryon \\
		Neutron & 939.566 & $0$ & $0$ & Baryon \\
		\midrule
		Pion ($\pi^+$) & 139.570 & $+1$ & $0$ & Meson \\
		Pion ($\pi^-$) & 139.570 & $-1$ & $0$ & Meson \\
		\midrule
		Kaon ($K^+$) & 493.677 & $+1$ & $\pm1$ & Meson \\
		Kaon ($K^-$) & 493.677 & $-1$ & $\pm1$ & Meson \\
		\midrule
		Lambda ($\Lambda^0$) & 1115.683 & $0$ & $-1$ & Baryon \\
		\midrule
		Sigma ($\Sigma^+$) & 1189.37 & $+1$ & $-1$ & Baryon \\
		Sigma ($\Sigma^-$) & 1197.45 & $-1$ & $-1$ & Baryon \\
		\midrule
		Xi ($\Xi^0$) & 1314.86 & $0$ & $-2$ & Baryon \\
		Xi ($\Xi^-$) & 1321.71 & $-1$ & $-2$ & Baryon \\
		\midrule
		Omega ($\Omega^-$) & 1672.45 & $-1$ & $-3$ & Baryon \\
		\midrule
		Electron & 0.511 & $-1$ &  & Lepton \\
		Neutrino & $<2.2 \times 10^{-9}$ & $0$ &  & Lepton \\
		\bottomrule
	\end{tabular}	
	\caption{Classification of Particle Strangeness}
\end{table}

All particles which are together with the neutrons and protons are called \keyword{baryons}, and the following ones exist: There is a ``lambda,'' with a mass of 1115 MeV, and three others, called sigmas, minus, neutral, and plus, with several masses almost the same. In addition to the baryons the other particles which are involved in the nuclear interaction are called \keyword{mesons}. There are first the \keyword{pions}, which come in three varieties, positive, negative, and neutral; they form another multiplet. We have also found some new things called K-mesons, and they occur as a doublet, $K^+$ and $K^-$. Also, every particle either has an antiparticle or is its own antiparticle. There are then \keyword{leptons}, which are particles which do not interact strongly in nuclei, have nothing to do with a nuclear interaction, and do not have a strong interaction. Finally, we have two other particles which do not interact strongly with the nuclear ones: one is a photon, and perhaps, if the field of gravity also has a quantum-mechanical analog (a quantum theory of gravitation has not yet been worked out), then there will be a particle, a \keyword{graviton}, which will have zero mass\footnote{The term ``zero mass'' in this context refers to the particles rest energy. If a particle has ``zero mass'' then it can never be found at rest. A photon for example, will always travel at a speed near $c$.}. 














