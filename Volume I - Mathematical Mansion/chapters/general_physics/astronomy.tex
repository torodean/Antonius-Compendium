\section{Astronomy}

\keyword{Astronomy} is the scientific study of celestial objects, phenomena, and the universe as a whole. It encompasses the observation, analysis, and understanding of stars, planets, galaxies, and other cosmic structures, as well as the exploration of space and the fundamental principles governing the cosmos. Astronomy plays a crucial role in unraveling the mysteries of the universe, from studying the origins of galaxies to investigating the properties of distant exoplanets and the evolution of the cosmos over billions of years. Astronomy is older than physics and bridged the way for the beginning of physics through observations of the simplistic motion of starts and planets. 

The most remarkable discovery in all of astronomy is that the stars are made of atoms of the same kind as those on the earth. With a \keyword{spectroscope}, we can analyze the frequencies of light waves and can therefore see the presence of atoms that are in different stars. Some chemicals were even discovered in stars before being discovered on earth.

One of the most impressive discoveries was the origin of energy in stars. The nuclear `burning' of hydrogen supplies energy to stars. The stellar properties inside stars make them prime conditions for many nuclear reactions to occur and new elements to be created. The isotopic proportions in our own composition reveal the stellar origins of the elements, suggesting that we were "cooked" in stars and expelled through novae and supernovae explosions. Astronomy, closely linked to physics, offers valuable insights into these processes.