\section{Chemistry}

\keyword{Inorganic chemistry} is the branch of chemistry that deals with the study of inorganic compounds, which are substances not primarily based on carbon-hydrogen (C-H) bonds. It explores the properties, structures, and behaviors of minerals, metals, gases, and other inorganic substances, playing a crucial role in understanding the fundamental elements and their interactions. The theory of chemistry and chemical reactions, was summarized to a large extent in the periodic chart of \keyword{Mendeleev}, also known as the \keyword{periodic table of elements}. All of these rules were eventually explained by quantum mechanics and thus theoretical chemistry is actually just physics. \keyword{Organic chemistry} is the chemistry of substances which are associated with living things. These are typically complex. This field is closely related to biology.

\keyword{Statistical mechanics} was a branch of physics developed alongside chemistry and has played an extremely important role in history. In any chemical situation, there are a large number of atoms all interacting. If all of these atoms could be analyzed at each collision and kept track of, the phenomenon of chemical reactions could be predicted. Statistical mechanics is the science of heat or thermodynamics which accomplishes this to some degree.