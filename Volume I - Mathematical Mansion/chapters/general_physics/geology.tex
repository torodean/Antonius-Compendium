\section{Geology}

\keyword{Geology}, also called \keyword{earth sciences}, is the scientific study of the Earth's structure, composition, and processes that have shaped its history. It investigates the formation of rocks, minerals, and geological features, such as mountains, volcanoes, and earthquakes, as well as the evolution of landscapes and the interactions between the Earth's materials and natural forces. Geology plays a crucial role in understanding Earth's past, present, and future, and it contributes to various fields, including resource exploration, environmental management, and hazard assessment. The question basic to geology is, what makes the earth the way it is?

Meteorology is the scientific study of the Earth's atmosphere and weather phenomena. It closely relates to physics because it involves the application of fundamental physical principles, such as thermodynamics, fluid dynamics, and radiation, to understand and predict atmospheric processes. By analyzing the behavior of air masses, temperature, humidity, pressure, and wind patterns, meteorologists use physics-based models and observations to forecast weather conditions and study atmospheric phenomena like hurricanes, tornadoes, and climate change. The interaction between meteorology and physics plays a crucial role in advancing our understanding of the Earth's atmosphere and its impact on weather patterns and climate dynamics.

While we have considerable knowledge about earthquake waves and the Earth's density distribution, physicists face challenges understanding matter's properties at the immense pressures within the Earth's core. Unlike conditions in stars, the complex mathematics involved in these extreme circumstances remains a hurdle to tackle. Additionally, determining the density and behavior of rocks at high pressure requires experimental investigations, as our theoretical understanding is limited.
