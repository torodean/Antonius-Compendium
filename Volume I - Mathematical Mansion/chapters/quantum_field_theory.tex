\chapter{Quantum Field Theory}
\thispagestyle{fancy}

\keyword{Quantum Field Theory} (QFT) is a theoretical framework that combines the principles of quantum mechanics with special relativity to describe the behavior of particles as quantized fields. It is one of the fundamental theories in modern theoretical physics and forms the basis for understanding particle interactions at both the microscopic and cosmological scales. Key aspects and principles of Quantum Field Theory include:

\begin{itemize}
	\item \keyword{Quantized Fields}: In QFT, particles are described not as individual discrete entities but as quantized fields that pervade all of space and time. Each particle type corresponds to a specific field, and particles are represented as excitations or quanta of these fields.

	\item \keyword{Lagrangian Formalism}: QFT employs a Lagrangian formalism to construct the equations of motion and interactions for the fields. The Lagrangian describes the dynamics and symmetries of the system and allows for the derivation of fundamental equations, such as the equations of motion and scattering amplitudes.

	\item \keyword{Creation and Annihilation Operators}: QFT introduces creation and annihilation operators to describe particle creation and annihilation processes. These operators act on the field states to generate or destroy particles and provide a mathematical framework for quantized states.

	\item \keyword{Feynman Diagrams}: Feynman diagrams are a visual tool used in QFT to represent particle interactions and scattering processes. They provide a pictorial representation of complex particle interactions and are instrumental in calculating scattering amplitudes.

	\item \keyword{Renormalization}: Similar to Quantum Electrodynamics (QED), QFT encounters infinities in certain calculations. Renormalization is a method to remove these infinities and obtain finite, meaningful predictions.

	\item \keyword{Gauge Theories}: QFT includes gauge theories, which describe interactions involving force-carrying particles (gauge bosons). Examples of gauge theories include Quantum Electrodynamics (QED) and Quantum Chromodynamics (QCD).

	\item \keyword{Quantum Electrodynamics} (QED): QED is a specific example of Quantum Field Theory that describes the electromagnetic force and its interactions with charged particles through quantized electromagnetic fields and photons.
\end{itemize}

Quantum Field Theory has proven to be highly successful and is an essential framework for describing and understanding the behavior of elementary particles and their interactions. It is the foundation for the Standard Model of particle physics, which provides a comprehensive description of the known elementary particles and their interactions, except for gravity. Additionally, QFT plays a crucial role in studying fundamental questions about the nature of matter, energy, and the universe at the most fundamental level.

\section{Quantum Vacuum}

The \keyword{quantum vacuum} represents a dynamic state of Quantum Field Theory (QFT). It is the lowest energy state (ground state) of all all quantum fields, pervaded by fluctuating energy and transient particles that constantly pop into and out of existence \cite{bib:Being, Nothingness, and the Quantum Vacuum}. This vacuum is due to the \keyword{Heisenbery uncertainty principal}, which implies there is a fundamental limit to the precision with which we can know the properties of the quantum vacuum, thus leading to intrinsic fluctuations in the field. In QFT, every particle is an excitation of an underlying field that fills space. The fluctuating vacuum field can be explained by introducing virtual particles which constantly appear and disappear, borrowing and returning energy from the underlying fields. This phenomenon underlies the \keyword{Casumir effect}, which is when a measureable force is created between two uncharged parallel plats in a vacuum.
