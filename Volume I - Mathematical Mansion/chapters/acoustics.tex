\chapter{Acoustics}
\thispagestyle{fancy}

\keyword{Acoustics} is the scientific study of sound and its transmission, production, and effects. It is a branch of physics and engineering that explores the properties of sound waves, their interactions with various mediums, and their impact on our environment and human perception. Acoustics plays a crucial role in our everyday lives, from the design of concert halls and the development of audio technology to the understanding of natural phenomena like earthquakes and the behavior of underwater sound.

At its core, \keyword{acoustics} seeks to unravel the mysteries of sound, a complex and omnipresent phenomenon that shapes our auditory experiences and contributes to our understanding of the world. Acoustics is not just the study of auditory sound though. At its core, acoustics is the study of vibrations propogating through matter, which in some ways is the study of everything and plays a critical role in the universe and matter interactions.

\section{Units}

The fundamental unit of sound change is called the \keyword{decibel} (dB). Perceived levels of sound change according to a logarithmic relation of the actual power levels. Decibels are calculated relative to a ratio of change from power $P_1$ watts to a power $P_2$ watts. The change in decibels is then
\begin{align}
dB = 10 \log_{10}\left(\frac{P_2}{P_1}\right)
\end{align}














\section{Underwater Acoustics}

Large bodies of water, such as the large ones covering Earth's surface do not allow for the propogation of radio and radar waves. Water, especially salt water is highly dissipative because it exhibits strong conductivity. This conductivity atteniates electromagnetic waves extremely rapidly, which limits their range and usefulness. Therefore, acoustic waves are the more practical way of carrying information underwater \cite{bib:An Introduction To Underwater Acoustics}. Acoustic waves have different transmission characteristics in water than they do in air. Some advantages in water are that they have a propogation speed or four to five times higher than in air, they reach higher levels, and they undergo less attentiation and therefore can propogate over larger distances\footnote{Sound propogation in the ocean can be observed at rangest of thousands of kilometers\cite{bib:An Introduction To Underwater Acoustics}.}. Some typical constraints include higher ambient noise levels, and more echos. Many species utilize use of sound waves in oceans such as whales and cataceans. 
