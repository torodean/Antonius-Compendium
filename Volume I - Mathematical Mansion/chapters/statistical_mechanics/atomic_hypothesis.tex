\section{The Atomic Hypothesis\label{section:The Atomic Hypothesis}}

The \keyword{atomic hypothesis} is a framework for how matter works on an atomic level.

\begin{quotationbox}
	``All things are made of atoms—little particles that move around in perpetual motion, attracting each other when they are a little distance apart, but repelling upon being squeezed into one another. \cite{bib:feynman lectures}''
\end{quotationbox}

Many questions can arise from this.

\begin{questions}
	\item Where do these atoms get their energy from?
	\item Where do these atoms get their inertia from?
	\item Do these atoms transfer energy among eachother?
	\item Are these atoms within some medium with which they can interact?
 	\item What causes the atomic perpetual motion? Could it be some sort of brownian motion within a field (such as the zero-point field)?
\end{questions}

Atoms are on the scale of Angstroms\footnote{An angstrom, symbolized as ``\AA``, is a unit of length used primarily in the fields of chemistry, physics, and nanotechnology to measure atomic and molecular distances, as well as the size of atoms and molecules. One angstrom is equal to 0.1 nanometers or $10^{-10}$ meters, representing a very small scale typically used to describe atomic and molecular dimensions.}. The jiggling motion between atoms and molecules is what's typically represented as heat. \keyword{Heat} is a form of energy that is transferred between objects or systems due to temperature differences.

\begin{interestnote}
	For \ce{H2O}, the distance between the center of the hydrogen atom and the center of the oxygen atom is 0.957 \AA, with a $105^\circ3'$ angle between the hydrogen atoms.
\end{interestnote}

\begin{defn}[Pressure]{defn:pressure}
	\keyword{Pressure} is a fundamental concept in physics and thermodynamics. It is defined as the force $F$ applied per unit area $A$ on a surface. Mathematically, pressure $P$ is expressed as 
	\begin{align}
		P=\frac{F}{A}.
	\end{align}
	Pressure is a scalar quantity and is typically measured in units of \keyword{pascals} (Pa) in the International System of Units (SI). 
\end{defn}

In everyday life, pressure is commonly encountered in various contexts, such as atmospheric pressure (the pressure exerted by the Earth's atmosphere), fluid pressure (in liquids and gases), and contact pressure (when an object is in contact with a surface). When a gas is slowly compressed, temperature increases and visa versa. In solids, as opposed to liquids, the atoms will arrange themselves into an array called a \keyword{crystalline array}. The positions are dependent on the other atoms within the array. The atoms still jiggle and bibraye and have an associated heat. They are vibrating `in place' rather than bouncing around with a somewhat free motion that would occur in liquids or a gas. \keyword{Melting} is the process by which the vibrations cause the atoms to shake apart from the crystalline array. At absolute zero temperature, the vibrations reach a minimum allowed state that atoms can have - but not zero. 

\begin{questions}
	\item Where do atoms get their minimum energy from?
	\item What determines the crystalline array of a material?
\end{questions}

\begin{interestnote}
	Helium is a special case for atoms. Helium does not freeze at absolute zero into a solid without the help of applied pressures. 
\end{interestnote}

\keyword{Air} consists almost entirely of nitrogen, oxygen, and water vapor. The remaining elements are a small amount of \ce{CO2}, Argon, Neon, Helium, Methane (\ce{CH4}), Krypton, etc. \keyword{Evaporation} occurs when air molecules hit water and `chip away' the \ce{H2O} molecules. When the molecules or atoms in air hit against hte surface of water, some of the energy and heat is transferred to the water molecules. This causes the vibration of those molecules to increase. In the case of those molecules having enough energy to break away from the rest of the water, the molecules become water vapor. This process essentially goes in both directions continuously until there is no water left as a liquid. The air around the water will also bounce into the water molecules and mix with it - creating gassed water. This is what allows for bubbles to exist. Given a small sealed container, this process would continue indefinitely. The higher energy atoms escape which in turn creates a cooling effect on the liquid itself because the overall temperature would be lowered. This is why blowing on something how will cool it - because the atoms in the air will interact and `shed' the higher energy atoms from it or gain heat upon their interaction. 

\keyword{Ions} are atoms that have either gained or lost electrons. In a salt crystal, there are chlorine ions (one extra electron) and sodium ions (one fewer electron). The ions stick together by \keyword{electrical attraction}. The electrons have a negative charge and the protons within the nucleus of the atoms has a positive charge. Because these ions have opposite charges, they attract one another.

\begin{questions}
	\item What gives matter its charge?
	\item Why are protons positively charged, electrons negatively charged, and neutrons neutral charged?
	\item Why is charge intrinsically a binary feature of reality? Can there be partial charges?
\end{questions}

Nature is doing the same thing during a \keyword{chemical process} and a \keyword{physical process}. These are when atomic partners are rearranged. Atoms prefer very specific partners and molecule formations which is a peculiar part of nature. One example is when hydrogen gas (\ce{H2}) and oxygen gas (\ce{O2}) react together in the presence of a spark or heat to produce water vapor (\ce{H2O}). The reaction is exothermic, meaning it releases energy in the form of heat. During the reaction, the hydrogen molecules (\ce{H2}) break apart into individual hydrogen atoms, and the oxygen molecules (\ce{O2}) break apart into individual oxygen atoms. Then, the hydrogen atoms combine with the oxygen atoms to form water molecules (\ce{H2O}). The resulting water vapor is in the gaseous state, and it can condense into liquid water or freeze into ice depending on the temperature and pressure conditions.

\begin{questions}
	\item Can/should heat be thought of as a wave or field?
	\item What median does heat travel through between atoms?
\end{questions}

\keyword{kinetic energy} is the energy of motion. When \keyword{burning} occurs, some atoms form easily with little energy, while others snap together with tremendous commotion - releasing energy to everything around them. \keyword{Flames} are when this energy is so enormous that light is generated rather than just heat.

Every substance is some type of arrangement of atoms. A \keyword{molecule} is simple an arrangement of atoms. The concept of a molecule is only approximate though and exists for a certain class of substances. Sometimes there is a natural grouping in a substances structure. \keyword{Organic chemistry} is essentially detective work that guesses and checks and uses mixtures of chemicals and atoms in order to determine where atoms are located - it is often incredibly accurate.

When determining if atoms really exist, ideas of how to detect them arise. One such idea is referred to as \keyword{Brownian motion}, which is the motion of a larger object due to the interactions with the smaller objects around it which are moving. For example, a basketball in a swimming pool will move as the water molecules around it collide with it. This is one method of detecting the presence of atoms.

