\chapter{Astrophysics}
\thispagestyle{fancy}

\keyword{Astrophysics} is a branch of astronomy that deals with the study of celestial objects, phenomena, and the physical processes that govern the universe. It combines principles of physics with observational data to understand the behavior, composition, and evolution of stars, galaxies, planets, and other cosmic entities. Key areas of study in astrophysics include:

\begin{itemize}
	\item \keyword{Stellar Astrophysics}: This field focuses on the properties and life cycles of stars. It examines stellar formation, nuclear fusion processes that power stars, their structure, and the various stages of stellar evolution, including the final fate of stars as supernovae, white dwarfs, neutron stars, or black holes.

	\item \keyword{Galactic Astrophysics}: Galactic astrophysics explores the structure, dynamics, and evolution of galaxies, which are vast collections of stars, gas, dust, and dark matter. It investigates the formation of galaxies, their interactions, and the supermassive black holes at their centers.

	\item \keyword{Cosmology}: Cosmology studies the large-scale properties and evolution of the universe as a whole. It addresses questions about the universe's origin, its expansion, the distribution of galaxies, dark matter, dark energy, and the ultimate fate of the cosmos.

	\item \keyword{Exoplanets and Planetary Systems}: This area examines planets outside our solar system (exoplanets) and investigates planetary systems' formation and dynamics. It aims to find potentially habitable exoplanets and understand the diversity of planetary systems in the universe.

	\item \keyword{High-Energy Astrophysics}: High-energy astrophysics focuses on cosmic phenomena that involve extreme conditions, such as black hole accretion disks, active galactic nuclei, gamma-ray bursts, and cosmic rays. It explores high-energy emissions and their impact on the surrounding space.

	\item \keyword{Astrobiology}: Astrobiology is an interdisciplinary field that combines aspects of astronomy, biology, and chemistry to study the potential for life beyond Earth. It seeks to understand the conditions required for life to exist on other planets and moons in our solar system and beyond.
\end{itemize}

Astrophysicists use various observational and theoretical tools, such as telescopes, space missions, computer simulations, and mathematical models, to unravel the mysteries of the cosmos. Their research not only expands our knowledge of the universe's workings but also addresses fundamental questions about our place in the cosmos and the potential existence of life elsewhere in the universe.


\section{Space Weather Prediction}

\textbf{Space Weather Prediction Center (SWPC):} The NOAA Space Weather Prediction Center, based in Boulder, Colorado, is the primary U.S. agency for monitoring and forecasting space weather events. It provides real-time data, alerts, and predictions on solar activity (e.g., flares, coronal mass ejections), geomagnetic storms, and ionospheric disturbances. SWPC uses satellite observations, ground-based magnetometers, and models to assess impacts on power grids, satellites, and communications. Key products include the Kp index, solar wind parameters, and auroral forecasts, accessible via \url{https://www.swpc.noaa.gov}.

\textbf{GloTEC (Global Total Electron Content):} GloTEC refers to global measurements of Total Electron Content (TEC), the number of free electrons in a column of the ionosphere (50--1000 km altitude), typically derived from GNSS (e.g., GPS) signal delays. TEC maps, produced by agencies like the International GNSS Service (IGS) or NOAA, reveal ionospheric dynamics, with sudden ``flashes'' (TEC spikes) indicating particle precipitation or disturbances. GloTEC data helps study space weather impacts on radio propagation and navigation systems, with units in TECU ($1\text{ TECU} = 10^{16}\text{electrons}/m^2$). GloTEC is accessible via \url{https://www.swpc.noaa.gov/products/glotec}
